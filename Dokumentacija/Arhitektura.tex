\chapter{Arhitektura i dizajn sustava}

\noindent Arhitektura je podijeljena na 2 dijela:
\begin{itemize}
  \item Web poslužitelj
  \item Baza podataka
\end{itemize}

\underline{Web preglednik} je "program koji omogućava surfanje". On služi kao korisničko sučelje i prikazivatelj podataka te enkapsulira kompletnu komunikaciju i HTTP zahtjeve te odgovore između korisnika i servera. Preglednici interpretiraju HTTP odgovore servera, najčešće HTML dokumente (i dodatke na koje se HTML dokumenti referiraju) te ih prikladno prikazuje.\hfill \break

\underline{Web poslužitelj} je program koji obrađuje podatke, čeka korisnike te interagira s njima (odgovara HTML dokumentima). 
Odabrali smo Express jer su svi članovi tima dobro upoznati s njim i nismo smatrali da je projekt suviše kompleksan za ovu (u svijetu raširenu) tehnologiju. U našem projektu on je također zadužen za komunikaciju s bazom podataka i obradu zahtjeva koje dobiva od preglednika. Obrada zahtjeva rezultira slanjem HTML-a pregledniku umjesto da se šalju JSON podaci. Razlog tome jest da želimo održati HATEOAS (Hypermedia as the Engine of Application State) princip zajedno sa REST (Representational State Transfer) principom. Zbog toga koristimo HTMX koji nam omogućava razvoj moderne i interaktivne aplikacije bez da kompromitiramo navedene principe. Dodatno, HTMX rješava probleme kao: mali broj elemenata koji mogu slati zahtjeve, podržanost isključivo GET i POST metoda, osvježavanje cijele stranice umjesto samo ažuriranih komponenti. Prednost ovog načina rada jest i taj da je svo stanje zapravo na serveru, a HTMX ga prikazuje. Tada ne moramo dodatno sinkronizirati stanje na serveru i stanje na klijentu i  drastično se smanjuje kompleksnost aplikacije.\hfill \break

\underline{Baza podataka} koristi se za pohranu, dohvaćanje, brisanje i ažuriranje podataka. Koristimo tehnologiju SQLite3 jer je jednostavna za korištenje, ne zahtijeva nikakvu konfiguraciju i dovoljno je brza za potrebe manjih do srednjih aplikacija.\hfill \break
\eject
U svrhu bolje organizacije koda, aplikacija je podijeljena na module. Pošto iz dokumentacije znamo što koji dio aplikacije radi, moguće je na svim modulima raditi aplikacije paralelno. Svaki se može interpretirati kao zasebna cjelina koja se sastoji od serverske tehnologije i baze podataka.\hfill \break

Dijelovi serverske strane aplikacije su:
\begin{itemize}
  \item Sloj domene (engl. \textit{routes})
  \item Sloj nadzora (engl. \textit{controllers})
  \item Sloj baze podataka (engl. \textit{database})
  \item Sloj podataka (engl. \textit{models})
\end{itemize}
\hfill
\\

\underline{Sloj domene} je sloj koji se sastoji od Express ruta. U ovom su sloju definirane rute koje se mogu pozvati iz korisničkog sučelja aplikacije te se u njima definira i koje funkcije iz sloja nadzora trebamo pozvati.\hfill \break

\underline{Sloj nadzora} je sloj koji se sastoji od Express kontrolera. Zadatak  ovog sloja jest obrada zahtjeva sloja domene. U ovom se sloju pozivaju funkcije koje koriste bazu podataka i upite iz sloja baze podataka.\hfill \break

\underline{Sloj podataka} koristi se za definiranje izgleda baze podataka. Ovaj se sloj koristi kada se baza prvi put stvara kako bi se automatski definirao izgled baze podataka (engl. \textit{migrate}).\hfill \break

\underline{Sloj baze podataka} je sloj koji se sastoji od upita prema bazi podataka. Odlučili smo ga odvojiti od sloja nadzora kako bi se izbjeglo dupliciranje koda te kako bismo imali što manje konflikata kod spajanja.\hfill \eject


Tijek dohvaćanja informacija iz baze podataka:
\begin{itemize}
	\item Sloj korisnika
	\item Sloj domene
	\item Sloj nadzora
	\item Sloj baze podataka
	\item Sloj podataka
  \end{itemize}

Sumiranje svih prednosti:
\begin{itemize}
  \item Jednostavnost "prednjeg" dijela sustava zbog HATEOAS i REST \newline principa (izbjegavanje dupliciranja stanja na klijentu)
  \item Jednostavnost nadogradnje i izmjene koda zbog odvojenosti slojeva
  \item Jednostavna baza podataka koja ne zahtijeva nikakvu konfiguraciju
  \item Jednostavna instalacija i pokretanje aplikacije (\textit{docker compose})
\end{itemize}

				
				\section{Baza podataka}

Za upravljanje podacima koristimo bazu podataka koja koristi SQLite, lagan i ugrađeni sustav za upravljanje bazama podataka koji se često koristi za manje aplikacije i mobilne aplikacije zbog svoje jednostavnosti i prenosivosti. SQLite omogućava učinkovito pohranjivanje i upravljanje podacima u lokalnom okruženju te je posebno prikladan za manje sustave koji ne zahtijevaju složene baze podataka. Sustav je temeljen na relacijskom modelu i sadrži sljedeće tablice:
\begin{itemize}
    \item user
    \item dictionary
    \item dictionary\_user
    \item word
    \item active\_question
    \item user\_word
    \item language
\end{itemize}

\subsection{Opis tablica}

Tablica \textbf{"user"} sadrži informacije o korisnicima sustava. Svaki korisnik ima jedinstveni identifikator (id), ime (name), prezime (surname), adresu elektroničke pošte (email), lozinku (password) te zastavicu is\_admin koja označava je li korisnik administrator (1) ili ne (0). Zastavica is\_admin je predstavljena INTEGER tipom jer SQLite ne podržava tip BOOLEAN.


\begin{longtblr}[
    label=none,
    entry=none
]{
    width = \textwidth,
    colspec={|X[6,l]|X[6, l]|X[20, l]|},
    rowhead = 1,
}
\hline \SetCell[c=3]{c}{\textbf{user}} \\ \hline[3pt]
\SetCell{LightGreen}id & INTEGER & Jedinstveni identifikator korisnika. \\ \hline
name & VARCHAR & Ime korisnika. \\ \hline
surname & VARCHAR & Prezime korisnika. \\ \hline
email & VARCHAR & Adresa elektroničke pošte korisnika. \\ \hline
password & VARCHAR & Lozinka korisnika. \\ \hline
is\_admin & INTEGER & Označava je li korisnik administrator (1/0). \\ \hline
token & VARCHAR & Token koji se koristi za autentifikaciju korisnika. \\ \hline
\end{longtblr}

Tablica \textbf{"dictionary"} pohranjuje podatke o rječnicima. Svaki rječnik ima jedinstveni identifikator (id), naziv rječnika (name) te informaciju o jeziku rječnika (language\_id).

\begin{longtblr}[
    label=none,
    entry=none
]{
    width = \textwidth,
    colspec={|X[6,l]|X[6, l]|X[20, l]|},
    rowhead = 1,
}
\hline \SetCell[c=3]{c}{\textbf{dictionary}} \\ \hline[3pt]
\SetCell{LightGreen}id & INTEGER & Jedinstveni identifikator rječnika. \\ \hline
name & VARCHAR & Naziv rječnika. \\ \hline
language\_id & VARCHAR & Referenca na jezik rječnika. \\ \hline
image\_link & VARCHAR & Link na sliku koja predstavlja rječnik. \\ \hline
\end{longtblr}

Tablica \textbf{"dictionary\_user"} uspostavlja povezanost između korisnika i rječnika, bolje rečeno sadrži informaciju koji su korisnici pretplaćeni na koje rječnike. Svaki zapis u ovoj tablici ima jedinstveni identifikator (id), referencu na korisnika (user\_id) i referencu na rječnik na koji je korisnik pretplaćen (dictionary\_id).

\begin{longtblr}[
    label=none,
    entry=none
]{
    width = \textwidth,
    colspec={|X[6,l]|X[6, l]|X[20, l]|},
    rowhead = 1,
}
\hline \SetCell[c=3]{c}{\textbf{dictionary\_user}} \\ \hline[3pt]
\SetCell{LightGreen}id & INTEGER & Jedinstveni identifikator zapisa. \\ \hline
user\_id & INTEGER & Referenca na korisnika koji je povezan s određenim rječnikom. \\ \hline
dictionary\_id & INTEGER & Referenca na rječnik koji je dodijeljen korisniku. \\ \hline
\end{longtblr}

Tablica \textbf{"word"} sadrži informacije o riječima koje korisnici uče. Svaka riječ ima jedinstveni identifikator (id) te informacije o stranoj riječi (foreign\_word), dodatnom opisu strane riječi (foreign\_description), prijevodu na hrvatski jezik (native\_word), dodatnom opisu hrvatskog prijevoda (native\_description), zvučnoj datoteci za izgovor riječi (pronunciation) te referencu na rječnik kojem pripada riječ (dictionary\_id).


\begin{longtblr}[
    label=none,
    entry=none
]{
    width = \textwidth,
    colspec={|X[10,l]|X[6, l]|X[20, l]|},
    rowhead = 1,
}
\hline \SetCell[c=3]{c}{\textbf{word}} \\ \hline[3pt]
\SetCell{LightGreen}id & INTEGER & Jedinstveni identifikator zapisa. \\ \hline
foreign\_word & VARCHAR & Strana riječ koju korisnici uče. \\ \hline
foreign\_description & VARCHAR & Dodatne informacije ili opis za stranu riječ. \\ \hline
native\_word & VARCHAR & Prijevod strane riječi na hrvatski jezik. \\ \hline
native\_description & VARCHAR & Dodatne informacije ili opis za hrvatski prijevod riječi. \\ \hline
pronunciation & VARCHAR & Zvučna datoteka koja sadrži izgovor strane riječi. \\ \hline
dictionary\_id & INTEGER & Referenca na rječnik kojem pripada riječ. \\ \hline
\end{longtblr}

Tablica \textbf{"active\_questions"} sadrži informacije o aktivnim pitanjima povezanim s određenim korisnicima. Koristi se u svrhu praćenja pitanja na različitim platformama. Svakom korisniku je dodijeljen maksimalno 1 redak u ovoj tablici. Svako aktivno pitanje ima jedinstveni identifikator (id) i referencu na riječ na koju se pitanje odnosi (word\_id), referencu na korisnika koji trenutno rješava pitanje (user\_id) te označava vrstu pitanja (1 za odabir ponuđenog odgovora, 2 za tipkanje odgovora, 3 za test izgovora) putem stupca type.


\begin{longtblr}[
    label=none,
    entry=none
]{
    width = \textwidth,
    colspec={|X[6,l]|X[6, l]|X[20, l]|},
    rowhead = 1,
}
\hline \SetCell[c=3]{c}{\textbf{active\_question}} \\ \hline[3pt]
\SetCell{LightGreen}id & INTEGER & Jedinstveni identifikator aktivnog pitanja. \\ \hline
word\_id & INTEGER & Referenca na riječ na koju se pitanje odnosi. \\ \hline
user\_id & INTEGER & Referenca na korisnika koji trenutno rješava pitanje. \\ \hline
type & INTEGER & Označava vrstu pitanja (1 za odabir ponuđenog odgovora, 2 za tipkanje odgovora, 3 za test izgovora). \\ \hline
\end{longtblr}

Tablica \textbf{"user\_word"} sadrži podatke o riječima koje su povezane s određenim korisnicima. Praktički, to je skupina riječi koje sustav može ispitati korisnika. Svaki zapis u tablici ima jedinstveni identifikator (id) te referencu na korisnika kojem riječ pripada (user\_id), referencu na riječ (word\_id), datum kad je korisnik zadnji put točno odgovorio na riječ (last\_answered), vremenski odmak za iduće pojavljivanje riječi (delay) te zastavicu active koja označava je li riječ trenutno aktivna za danog korisnika (1) ili ne (0).


\begin{longtblr}[
    label=none,
    entry=none
]{
    width = \textwidth,
    colspec={|X[6,l]|X[6, l]|X[20, l]|},
    rowhead = 1,
}
\hline \SetCell[c=3]{c}{\textbf{user\_word}} \\ \hline[3pt]
\SetCell{LightGreen}id & INTEGER & Jedinstveni identifikator zapisa. \\ \hline
user\_id & INTEGER & Referenca na korisnika kojem riječ pripada. \\ \hline
word\_id & INTEGER & Referenca riječi. \\ \hline
last\_answered & VARCHAR & Datum kad je korisnik zadnji put točno odgovorio na riječ. \\ \hline
delay & INTEGER & Vremenski odmak za iduće pojavljivanje riječi. \\ \hline
active & INTEGER & Je li riječ trenutno aktivna za danog korisnika (1/0). \\ \hline
\end{longtblr}

Tablica \textbf{"language"} sadrži podatke o jeziku koji se riječnikom uči. Svaki zapis u tablici ima jedinstveni identifikator (id), naziv jezika (name), skraćenicu jezika (shorthand), te ikonicu (flag\_icon).

\begin{longtblr}[
    label=none,
    entry=none
]{
    width = \textwidth,
    colspec={|X[6,l]|X[6, l]|X[20, l]|},
    rowhead = 1,
}
\hline \SetCell[c=3]{c}{\textbf{language}} \\ \hline[3pt]
\SetCell{LightGreen}id & INTEGER & Jedinstveni identifikator zapisa. \\ \hline
name & VARCHAR & Ime jezika. \\ \hline
shorthand & VARCHAR & Skraćenica jezika. \\ \hline
flag\_icon & VARCHAR & Ikonica jezika. \\ \hline
\end{longtblr}

					
			\subsection{Dijagram baze podataka}
				\begin{figure}[H]
					\includegraphics[scale=0.5]{dijagrami/baza_dijagram.png} 
					\centering
					\caption{Dijagram razreda}
					\label{fig:class_diagram}
				\end{figure}			
			\eject
			
			
		\section{Dijagram razreda}        

			
			%\subsection{Dijagram kontrolera}
            Dijagram razreda se koristi za prikaz razreda sustava, njihovih atributa, metoda i odnosa između njih.
            Zbog preglednosti, dijagram je podijeljen na 3 dijela:
            \begin{packed_item}
                \item \textbf{Kontroleri}
                \item \textbf{Rute}
                \item \textbf{Modeli}
            \end{packed_item}
           

            Slika 4.2 je dijagram kontrolera koji pokazuje dostupne funkcije i njihove atribute te povratne vrijednosti. Također prikazuje koji routeri koriste te funkcije.
				\begin{figure}[H]
					\includegraphics[width=0.9\textwidth]{dijagrami/slika1.png} 
					\centering
					\caption{Dijagram kontrolera}
					\label{fig:class_diagram}
				\end{figure}			
			\eject
            Dijagram ruta na slici 4.3 prikazuje koje su sve rute dostupne i modele koje koriste iz baze podataka. U projektu postoje dva tipa ruta. Prvi tip ruta je zadužen samo za slanje statičnih stranica u HTML-u. Drugi tip izvodi napredne operacije na bazi podataka (npr. Dohvati sve riječi iz rječnika), zatim generira HTML koristeći te podatke te šalje korisniku u HTML formatu.
           % \subsection{Dijagram Ruta}
				\begin{figure}[H]
					\includegraphics[width=0.9\textwidth]{dijagrami/slika2.png} 
					\centering
					\caption{Dijagram ruta}
					\label{fig:class_diagram}
				\end{figure}			
			\eject
            Slika 4.4 prikazuje sve modele koji se koriste u projektu. Kod MVC projekata, u ovom dijelu su često neke klase nad kojoma se rade neke operacije. Nakon što se ti podaci promijene u toj klasi se onda promjene šalju na bazu podataka.
            Slična stvar postoji i u javascript-u (node.js) i nazivaju se ORM (Object Relational Mapping) alati. ORM alati omogućuju da se podaci iz baze podataka prikažu kao objekti u kodu te da se ti objekti mogu jednostavno mijenjati i spremati u bazu podataka \textit{bez potrebe pisanja SQL upita}.
            Problem kod implementacije svih ORM-a je da značajno usporavaju rad aplikacije zbog mnogih slojeva apstrakcije. Svi članovi dobro znaju SQL upite i pisanje SQL-a bi bilo jednako brzo kao i korištenje ORM-a. Zbog toga smo se odlučili za pisanje čistih SQL upita bez nepotrebne apstrakcije.
            %\subsection{Dijagram Modela}
				\begin{figure}[H]
					\includegraphics[width=0.9\textwidth]{dijagrami/slika3.jpg} 
					\centering
					\caption{Dijagram modela}
					\label{fig:class_diagram}
				\end{figure}			
			\eject
			
		\section{Dijagram stanja}
			
			
        \begin{figure}[H]
            \includegraphics[width=1.1\textwidth]{dijagrami/Dijagram_stanja.png} 
            \centering
            \caption{Dijagram modela}
            \label{fig:class_diagram}
        \end{figure}			
			
			\eject 
		
		\section{Dijagram aktivnosti}
			
        \begin{figure}[H]
            \includegraphics[width=0.6\textwidth]{dijagrami/Dijagram Aktivnosti.png} 
            \centering
            \caption{Dijagram modela}
            \label{fig:class_diagram}
        \end{figure}	

			\eject
		\section{Dijagram komponenti}
		
			