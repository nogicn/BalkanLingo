\chapter{Arhitektura i dizajn sustava}
		
		\textbf{\textit{dio 1. revizije}}\\

\noindent Arhitektura je podijeljena na 2 dijela:
\begin{itemize}
  \item Web poslužitelj
  \item Baza podataka
\end{itemize}

\underline{Web preglednik} je program koji služi za prikaz web stranica. Svaki preglednik interpretira HTML dokumente i prikazuje ih korisniku. On je zapravo posrednik između korisnika i podataka kojima želi pristupiti.\hfill \break

\underline{Web poslužitelj} je program koji šalje HTML dokumente pregledniku. 
Odabrali smo Express jer su svi već upoznati s njim sa predmeta web1. 
U našem projektu on je također zadužen za komunikaciju s bazom podataka i obradu zahtjeva koje dobiva od preglednika. Obrada zahtjeva rezultira slanjem HTML-a pregledniku umjesto da šalje JSON podatke. Razlog toga je da želimo održati HATEOAS (Hypermedia as the Engine of Application State) princip zajedno sa REST (Representational State Transfer) principom. Upravo zbog toga koristimo HTMX library, koji nam omogućava da dobijemo modernu interaktivnu aplikaciju, ali bez da izgubimo HATEOAS i REST principe. Prednost ovog načina rada je da je svo stanje na serveru, 
dakle ima samo jedan izvor istine što drastično smanjuje kompleksnost aplikacije.\hfill \break

\underline{Baza podataka} se koristi za pohranjivanje, dohvaćanje, brisanje i ažuriranje podataka. Za bazu smo odlučili koristiti SQLite3 jer je jednostavna za korištenje, ne zahtjeva nikakvu konfiguraciju i dovoljno je brza za potrebe manjih do srednjih aplikacija.\hfill \break

Radi bolje organizacije koda, aplikacija je podijeljena na module. Pošto znamo iz dokumentacije što koji dio aplikacije radi i što je potrebno za unutarnju komunikaciju moguće je raditi sve dijelove aplikacije paralelno. Svaki modul je zasebna cjelina koja se sastoji od Expressa i baze podataka.\hfill \break

Dijelovi backend aplikacije na web poslužitelju su:
\begin{itemize}
  \item Sloj domene (engl. routes)
  \item Sloj nadzora (engl. controllers)
  \item Sloj baze podataka (engl. database)
  \item Sloj podataka (engl. models)
\end{itemize}
\hfill \break
\underline{Sloj domene} je sloj koji se sastoji od express ruta. U ovom sloju su definirane rute koje se mogu pozvati iz React aplikacije te se u njima definiraju koje funkcije iz sloja nadzora se trebaju pozvati.\hfill \break

\underline{Sloj nadzora} je sloj koji se sastoji od express kontrolera. Njegov zadatak je da obradi zahtjev koji je dobio od sloja domene. U ovom sloju se pozivaju funkcije koje koriste upite iz sloja baze podataka i bazu podataka.\hfill \break

\underline{Sloj podataka} se koristi za definiranje izgleda baze podataka. Ovaj sloj se koristi kada se baza prvi put stvara kako bi se automatski definirao izgled baze podataka (engl. migrate).\hfill \break

\underline{Sloj baze podataka} je sloj koji se sastoji od upita prema bazi podataka. Odlučili smo ga odvojiti od sloja nadzora kako bi se izbjeglo dupliciranje koda te kako bi imali što manje konflikata kod spajanja.\hfill \break


Tijek dohvaćanja informacija iz baze podataka:
\begin{itemize}
	\item Sloj korisnika
	\item sloj domene
	\item sloj nadzora
	\item sloj baze podataka
	\item sloj podataka
  \end{itemize}

Sumiranje svih prednosti:
\begin{itemize}
  \item Jednostavnost "prednjeg" dijela sustava zbog HATEOAS i REST principa (izbjegavanje dupliciranja stanja na klijentu)
  \item Jednostavnost produljenja i izmjene koda zbog odvojenosti slojeva
  \item Jednostavna baza podataka koja ne zahtjeva nikakvu konfiguraciju
  \item Jednostavna instalacija i pokretanje aplikacije (docker compose)
\end{itemize}


	
		

		

				
		\section{Baza podataka}
			
			\textbf{\textit{dio 1. revizije}}\\
			
		\textit{Potrebno je opisati koju vrstu i implementaciju baze podataka ste odabrali, glavne komponente od kojih se sastoji i slično.}
		
			\subsection{Opis tablica}
			

				\textit{Svaku tablicu je potrebno opisati po zadanom predlošku. Lijevo se nalazi točno ime varijable u bazi podataka, u sredini se nalazi tip podataka, a desno se nalazi opis varijable. Svjetlozelenom bojom označite primarni ključ. Svjetlo plavom označite strani ključ}
				
				
				\begin{longtblr}[
					label=none,
					entry=none
					]{
						width = \textwidth,
						colspec={|X[6,l]|X[6, l]|X[20, l]|}, 
						rowhead = 1,
					} %definicija širine tablice, širine stupaca, poravnanje i broja redaka naslova tablice
					\hline \SetCell[c=3]{c}{\textbf{korisnik - ime tablice}}	 \\ \hline[3pt]
					\SetCell{LightGreen}IDKorisnik & INT	&  	Lorem ipsum dolor sit amet, consectetur adipiscing elit, sed do eiusmod  	\\ \hline
					korisnickoIme	& VARCHAR &   	\\ \hline 
					email & VARCHAR &   \\ \hline 
					ime & VARCHAR	&  		\\ \hline 
					\SetCell{LightBlue} primjer	& VARCHAR &   	\\ \hline 
				\end{longtblr}
				
				
			
			\subsection{Dijagram baze podataka}
				\textit{ U ovom potpoglavlju potrebno je umetnuti dijagram baze podataka. Primarni i strani ključevi moraju biti označeni, a tablice povezane. Bazu podataka je potrebno normalizirati. Podsjetite se kolegija "Baze podataka".}
			
			\eject
			
			
		\section{Dijagram razreda}
		
			\textit{Potrebno je priložiti dijagram razreda s pripadajućim opisom. Zbog preglednosti je moguće dijagram razlomiti na više njih, ali moraju biti grupirani prema sličnim razinama apstrakcije i srodnim funkcionalnostima.}\\
			
			\textbf{\textit{dio 1. revizije}}\\
			
			\textit{Prilikom prve predaje projekta, potrebno je priložiti potpuno razrađen dijagram razreda vezan uz \textbf{generičku funkcionalnost} sustava. Ostale funkcionalnosti trebaju biti idejno razrađene u dijagramu sa sljedećim komponentama: nazivi razreda, nazivi metoda i vrste pristupa metodama (npr. javni, zaštićeni), nazivi atributa razreda, veze i odnosi između razreda.}\\
			
			\textbf{\textit{dio 2. revizije}}\\			
			
			\textit{Prilikom druge predaje projekta dijagram razreda i opisi moraju odgovarati stvarnom stanju implementacije}
			
			
			
			\eject
		
		\section{Dijagram stanja}
			
			
			\textbf{\textit{dio 2. revizije}}\\
			
			\textit{Potrebno je priložiti dijagram stanja i opisati ga. Dovoljan je jedan dijagram stanja koji prikazuje \textbf{značajan dio funkcionalnosti} sustava. Na primjer, stanja korisničkog sučelja i tijek korištenja neke ključne funkcionalnosti jesu značajan dio sustava, a registracija i prijava nisu. }
			
			
			\eject 
		
		\section{Dijagram aktivnosti}
			
			\textbf{\textit{dio 2. revizije}}\\
			
			 \textit{Potrebno je priložiti dijagram aktivnosti s pripadajućim opisom. Dijagram aktivnosti treba prikazivati značajan dio sustava.}
			
			\eject
		\section{Dijagram komponenti}
		
			\textbf{\textit{dio 2. revizije}}\\
		
			 \textit{Potrebno je priložiti dijagram komponenti s pripadajućim opisom. Dijagram komponenti treba prikazivati strukturu cijele aplikacije.}