\chapter{Arhitektura i dizajn sustava}
		
		\textbf{\textit{dio 1. revizije}}\\

\noindent Arhitektura je podijeljena na 2 dijela:
\begin{itemize}
  \item Web poslužitelj
  \item Baza podataka
\end{itemize}

\underline{Web preglednik} je program koji služi za prikaz web stranica. Svaki preglednik interpretira HTML dokumente i prikazuje ih korisniku. On je zapravo posrednik između korisnika i podataka kojima želi pristupiti.\hfill \break

\underline{Web poslužitelj} je program koji šalje HTML dokumente pregledniku. 
Odabrali smo Express jer su svi već upoznati s njim sa predmeta web1. 
U našem projektu on je također zadužen za komunikaciju s bazom podataka i obradu zahtjeva koje dobiva od preglednika. Obrada zahtjeva rezultira slanjem HTML-a pregledniku umjesto da šalje JSON podatke. Razlog toga je da želimo održati HATEOAS (Hypermedia as the Engine of Application State) princip zajedno sa REST (Representational State Transfer) principom. Upravo zbog toga koristimo HTMX library, koji nam omogućava da dobijemo modernu interaktivnu aplikaciju, ali bez da izgubimo HATEOAS i REST principe. Prednost ovog načina rada je da je svo stanje na serveru, 
dakle ima samo jedan izvor istine što drastično smanjuje kompleksnost aplikacije.\hfill \break

\underline{Baza podataka} se koristi za pohranjivanje, dohvaćanje, brisanje i ažuriranje podataka. Za bazu smo odlučili koristiti SQLite3 jer je jednostavna za korištenje, ne zahtjeva nikakvu konfiguraciju i dovoljno je brza za potrebe manjih do srednjih aplikacija.\hfill \break

Radi bolje organizacije koda, aplikacija je podijeljena na module. Pošto znamo iz dokumentacije što koji dio aplikacije radi i što je potrebno za unutarnju komunikaciju moguće je raditi sve dijelove aplikacije paralelno. Svaki modul je zasebna cjelina koja se sastoji od Expressa i baze podataka.\hfill \break

Dijelovi backend aplikacije na web poslužitelju su:
\begin{itemize}
  \item Sloj domene (engl. routes)
  \item Sloj nadzora (engl. controllers)
  \item Sloj baze podataka (engl. database)
  \item Sloj podataka (engl. models)
\end{itemize}
\hfill \break
\underline{Sloj domene} je sloj koji se sastoji od express ruta. U ovom sloju su definirane rute koje se mogu pozvati iz React aplikacije te se u njima definiraju koje funkcije iz sloja nadzora se trebaju pozvati.\hfill \break

\underline{Sloj nadzora} je sloj koji se sastoji od express kontrolera. Njegov zadatak je da obradi zahtjev koji je dobio od sloja domene. U ovom sloju se pozivaju funkcije koje koriste upite iz sloja baze podataka i bazu podataka.\hfill \break

\underline{Sloj podataka} se koristi za definiranje izgleda baze podataka. Ovaj sloj se koristi kada se baza prvi put stvara kako bi se automatski definirao izgled baze podataka (engl. migrate).\hfill \break

\underline{Sloj baze podataka} je sloj koji se sastoji od upita prema bazi podataka. Odlučili smo ga odvojiti od sloja nadzora kako bi se izbjeglo dupliciranje koda te kako bi imali što manje konflikata kod spajanja.\hfill \break


Tijek dohvaćanja informacija iz baze podataka:
\begin{itemize}
	\item Sloj korisnika
	\item sloj domene
	\item sloj nadzora
	\item sloj baze podataka
	\item sloj podataka
  \end{itemize}

Sumiranje svih prednosti:
\begin{itemize}
  \item Jednostavnost "prednjeg" dijela sustava zbog HATEOAS i REST principa (izbjegavanje dupliciranja stanja na klijentu)
  \item Jednostavnost produljenja i izmjene koda zbog odvojenosti slojeva
  \item Jednostavna baza podataka koja ne zahtjeva nikakvu konfiguraciju
  \item Jednostavna instalacija i pokretanje aplikacije (docker compose)
\end{itemize}

				
				\section{Baza podataka}

Baza podataka
Za upravljanje podacima koristimo bazu podataka koja koristi SQLite, lagan i ugrađeni sustav za upravljanje bazama podataka koji se često koristi za manje aplikacije i mobilne aplikacije zbog svoje jednostavnosti i portabilnosti. SQLite omogućava učinkovito pohranjivanje i upravljanje podacima u lokalnom okruženju te je posebno prikladan za manje sustave koji ne zahtijevaju složene baze podataka. Sustav je temeljen na relacijskom modelu i sadrži slijedeće tablice.
\begin{itemize}
    \item user
    \item dictionary
    \item dictionary\_user
    \item word
    \item active\_question
    \item user\_word
\end{itemize}

\subsection{Opis tablica}

Tablica \textbf{"user"} sadrži informacije o korisnicima sustava. Svaki korisnik ima jedinstveni identifikator (id), ime (name), prezime (surname), email adresu (email), lozinku (password) te zastavicu is\_admin koja označava je li korisnik administrator (1) ili ne (0). Zastavica is\_admin je predstavljena INTEGER tipom jer SQLite ne podržava tip BOOLEAN.


\begin{longtblr}[
    label=none,
    entry=none
]{
    width = \textwidth,
    colspec={|X[6,l]|X[6, l]|X[20, l]|},
    rowhead = 1,
}
\hline \SetCell[c=3]{c}{\textbf{user}} \\ \hline[3pt]
\SetCell{LightGreen}id & INTEGER & Jedinstveni identifikator korisnika. \\ \hline
name & VARCHAR & Ime korisnika. \\ \hline
surname & VARCHAR & Prezime korisnika. \\ \hline
email & VARCHAR & Email adresa korisnika. \\ \hline
password & VARCHAR & Lozinka korisnika. \\ \hline
is\_admin & INTEGER & Označava da li je korisnik administrator (1/0). \\ \hline
\end{longtblr}

Tablica \textbf{"dictionary"} pohranjuje podatke o rječnicima. Svaki rječnik ima jedinstveni identifikator (id), naziv rječnika (name) te informaciju o jeziku rječnika (language).

\begin{longtblr}[
    label=none,
    entry=none
]{
    width = \textwidth,
    colspec={|X[6,l]|X[6, l]|X[20, l]|},
    rowhead = 1,
}
\hline \SetCell[c=3]{c}{\textbf{dictionary}} \\ \hline[3pt]
\SetCell{LightGreen}id & INTEGER & Jedinstveni identifikator rječnika. \\ \hline
name & VARCHAR & Naziv rječnika. \\ \hline
language & VARCHAR & Jezik riječnika. \\ \hline
\end{longtblr}

Tablica \textbf{"dictionary\_user"} uspostavlja povezanost između korisnika i rječnika, bolje rečeno sadrži informaciju koji su korisnici pretplaćeni na koje riječnike. Svaki zapis u ovoj tablici ima jedinstveni identifikator (id), referencu na korisnika (user\_id) i referencu na rječnik na koji je korisnik pretplaćen (dictionary\_id).

\begin{longtblr}[
    label=none,
    entry=none
]{
    width = \textwidth,
    colspec={|X[6,l]|X[6, l]|X[20, l]|},
    rowhead = 1,
}
\hline \SetCell[c=3]{c}{\textbf{dictionary\_user}} \\ \hline[3pt]
\SetCell{LightGreen}id & INTEGER & Jedinstveni identifikator zapisa. \\ \hline
user\_id & INTEGER & Referenca na korisnika koji je povezan s određenim rječnikom. \\ \hline
dictionary\_id & INTEGER & Referenca na rječnik koji je dodijeljen korisniku. \\ \hline
\end{longtblr}

Tablica \textbf{"word"} sadrži informacije o riječima koje korisnici uče. Svaka riječ ima jedinstveni identifikator (id) te informacije o stranoj riječi (foreign\_word), dodatnom opisu strane riječi (foreign\_description), prijevodu na hrvatski jezik (native\_word), dodatnom opisu hrvatskog prijevoda (native\_description), zvučnoj datoteci za izgovor riječi (pronunciation) te referencu na rječnik kojem pripada riječ (dictionary\_id).


\begin{longtblr}[
    label=none,
    entry=none
]{
    width = \textwidth,
    colspec={|X[10,l]|X[6, l]|X[20, l]|},
    rowhead = 1,
}
\hline \SetCell[c=3]{c}{\textbf{word}} \\ \hline[3pt]
\SetCell{LightGreen}id & INTEGER & Jedinstveni identifikator zapisa. \\ \hline
foreign\_word & VARCHAR & Strana riječ koju korisnici uče. \\ \hline
foreign\_description & VARCHAR & Dodatne informacije ili opis za stranu riječ. \\ \hline
native\_word & VARCHAR & Prijevod strane riječi na hrvatski jezik. \\ \hline
native\_description & VARCHAR & Dodatne informacije ili opis za hrvatski prijevod riječi. \\ \hline
pronunciation & VARCHAR & Zvučna datoteka koja sadrži izgovor strane riječi. \\ \hline
dictionary\_id & INTEGER & Referenca na rječnik kojem pripada riječ. \\ \hline
\end{longtblr}

Tablica \textbf{"active\_questions"} sadrži informacije o aktivnim pitanjima povezanim s određenim korisnicima. Koristi se u svrhu praćenja pitanja na različitim platformama. Svakom korisniku je dodijeljen maksimalno 1 redak u ovoj tablici. Svako aktivno pitanje ima jedinstveni identifikator (id) i referencu na riječ na koju se pitanje odnosi (word\_id), referencu na korisnika koji trenutno rješava pitanje (user\_id) te označava vrstu pitanja (1 za odabir ponuđenog odgovora, 2 za tipkanje odgovora, 3 za test izgovora) putem stupca type.


\begin{longtblr}[
    label=none,
    entry=none
]{
    width = \textwidth,
    colspec={|X[6,l]|X[6, l]|X[20, l]|},
    rowhead = 1,
}
\hline \SetCell[c=3]{c}{\textbf{active\_question}} \\ \hline[3pt]
\SetCell{LightGreen}id & INTEGER & Jedinstveni identifikator aktivnog pitanja. \\ \hline
word\_id & INTEGER & Referenca na riječ na koju se pitanje odnosi. \\ \hline
user\_id & INTEGER & Referenca na korisnika koji trenutno rješava pitanje. \\ \hline
type & INTEGER & Označava vrstu pitanja (1 za odabir ponuđenog odgovora, 2 za tipkanje odgovora, 3 za test izgovora). \\ \hline
\end{longtblr}

Tablica \textbf{"user\_word"} sadrži podatke o riječima koje su povezane s određenim korisnicima. Praktički, to je skupina riječi koje sustav može ispitati korisnika. Svaki zapis u tablici ima jedinstveni identifikator (id) te referencu na korisnika kojem riječ pripada (user\_id), referencu na riječ (word\_id), datum kad je korisnik zadnji put točno odgovorio na riječ (last\_answered), vremenski odmak za iduće pojavljivanje riječi (delay) te zastavicu active koja označava je li riječ trenutno aktivna za danog korisnika (1) ili ne (0).


\begin{longtblr}[
    label=none,
    entry=none
]{
    width = \textwidth,
    colspec={|X[6,l]|X[6, l]|X[20, l]|},
    rowhead = 1,
}
\hline \SetCell[c=3]{c}{\textbf{user\_word}} \\ \hline[3pt]
\SetCell{LightGreen}id & INTEGER & Jedinstveni identifikator zapisa. \\ \hline
user\_id & INTEGER & Referenca na korisnika kojem riječ pripada. \\ \hline
word\_id & INTEGER & Referenca riječi. \\ \hline
last\_answered & VARCHAR & Datum kad je korisnik zadnji put točno odgovorio na riječ. \\ \hline
delay & INTEGER & Vremenski odmak za iduće pojavljivanje riječi. \\ \hline
active & INTEGER & Da li je riječ trenutno aktivna za danog korisnika (1/0). \\ \hline
\end{longtblr}

					
			\subsection{Dijagram baze podataka}
				\begin{figure}[H]
					\includegraphics[scale=0.5]{dijagrami/baza_dijagram.png} 
					\centering
					\caption{Dijagram razreda}
					\label{fig:class_diagram}
				\end{figure}			
			\eject
			
			
		\section{Dijagram razreda}
		
			\textit{Potrebno je priložiti dijagram razreda s pripadajućim opisom. Zbog preglednosti je moguće dijagram razlomiti na više njih, ali moraju biti grupirani prema sličnim razinama apstrakcije i srodnim funkcionalnostima.}\\
			
			\textbf{\textit{dio 1. revizije}}\\
			
			\textit{Prilikom prve predaje projekta, potrebno je priložiti potpuno razrađen dijagram razreda vezan uz \textbf{generičku funkcionalnost} sustava. Ostale funkcionalnosti trebaju biti idejno razrađene u dijagramu sa sljedećim komponentama: nazivi razreda, nazivi metoda i vrste pristupa metodama (npr. javni, zaštićeni), nazivi atributa razreda, veze i odnosi između razreda.}\\
			
			\textbf{\textit{dio 2. revizije}}\\			
			
			\textit{Prilikom druge predaje projekta dijagram razreda i opisi moraju odgovarati stvarnom stanju implementacije}
			
			
			
			\eject
		
		\section{Dijagram stanja}
			
			
			\textbf{\textit{dio 2. revizije}}\\
			
			\textit{Potrebno je priložiti dijagram stanja i opisati ga. Dovoljan je jedan dijagram stanja koji prikazuje \textbf{značajan dio funkcionalnosti} sustava. Na primjer, stanja korisničkog sučelja i tijek korištenja neke ključne funkcionalnosti jesu značajan dio sustava, a registracija i prijava nisu. }
			
			
			\eject 
		
		\section{Dijagram aktivnosti}
			
			\textbf{\textit{dio 2. revizije}}\\
			
			 \textit{Potrebno je priložiti dijagram aktivnosti s pripadajućim opisom. Dijagram aktivnosti treba prikazivati značajan dio sustava.}
			
			\eject
		\section{Dijagram komponenti}
		
			\textbf{\textit{dio 2. revizije}}\\
		
			 \textit{Potrebno je priložiti dijagram komponenti s pripadajućim opisom. Dijagram komponenti treba prikazivati strukturu cijele aplikacije.}