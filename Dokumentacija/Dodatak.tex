\chapter*{Dodatak: Prikaz aktivnosti grupe}
		\addcontentsline{toc}{chapter}{Dodatak: Prikaz aktivnosti grupe}
		
		\section*{Dnevnik sastajanja}
		
		\textbf{\textit{Kontinuirano osvježavanje}}\\
		
		 \textit{U ovom dijelu potrebno je redovito osvježavati dnevnik sastajanja prema predlošku.}
		
		\begin{packed_enum}
			\item  sastanak
			
			\item[] \begin{packed_item}
				\item Datum: 17. listopada 2023.
				\item Prisustvovali: Svi
				\item Teme sastanka:
				\begin{packed_item}
					\item  Upoznavanje članova tima
					\item  Dogovr o načinu rada i okvirno o raspodjeli poslova
				\end{packed_item}
			\end{packed_item}
			
			\item  sastanak
			\item[] \begin{packed_item}
				\item Datum: u ovom formatu: 28. listopada 2023.
				\item Prisustvovali: Svi
				\item Teme sastanka:
				\begin{packed_item}
					\item  Određivanje izgleda i funkcionalnosti aplikacije
					\item  Započinjanje rada na dokumentaciji (UC)
				\end{packed_item}
			\end{packed_item}

			\item  sastanak
			\item[] \begin{packed_item}
				\item Datum: u ovom formatu: 31. listopada 2023.
				\item Prisustvovali: Svi
				\item Teme sastanka:
				\begin{packed_item}
					\item  Dovršavanje dokumentacije (UC)
					\item  Dopunjavanje dokumentacije baze podataka i arhitekture
				\end{packed_item}
			\end{packed_item}
			
			%
			
		\end{packed_enum}
		
		\eject
		\section*{Tablica aktivnosti}
		
			\textbf{\textit{Kontinuirano osvježavanje}}\\
			
			 \textit{Napomena: Doprinose u aktivnostima treba navesti u satima po članovima grupe po aktivnosti.}

			\begin{longtblr}[
					label=none,
				]{
					vlines,hlines,
					width = \textwidth,
					colspec={X[7, l]X[1, c]X[1, c]X[1, c]X[1, c]X[1, c]X[1, c]X[1, c]}, 
					vline{1} = {1}{text=\clap{}},
					hline{1} = {1}{text=\clap{}},
					rowhead = 1,
				} 
			
				\SetCell[c=1]{c}{} & \SetCell[c=1]{c}{\rotatebox{90}{\textbf{Nino Nogić}}} & \SetCell[c=1]{c}{\rotatebox{90}{\textbf{Hrvoje Biloš }}} &	\SetCell[c=1]{c}{\rotatebox{90}{\textbf{Borna Krušlin }}} & \SetCell[c=1]{c}{\rotatebox{90}{\textbf{Petar Pandža }}} &	\SetCell[c=1]{c}{\rotatebox{90}{\textbf{Martin Bogoje }}} & \SetCell[c=1]{c}{\rotatebox{90}{\textbf{Alberto Kerim }}} &	\SetCell[c=1]{c}{\rotatebox{90}{\textbf{Lana Kuretić }}} \\  
				Upravljanje projektom 		& 10 &  &  &  &  &  1 \\ 
				Opis projektnog zadatka 	& 3 &  &  &  &  & 1 & 4 \\ 
				
				Funkcionalni zahtjevi       & 1 &  &  &  &  & 3 & 1 \\ 
				Opis pojedinih obrazaca 	& 2 & 5 & 5 &  & 5 & 1 &  \\ 
				Dijagram obrazaca 			&  &  & 4 &  & 4 & &  \\ 
				Sekvencijski dijagrami 		&  & 4 & 2 &  & 2 & 4 &  \\ 
				Opis ostalih zahtjeva 		& 2 &  &  &  &  & 1 & 2  \\ 

				Arhitektura i dizajn sustava	 & 4 &  &  &  &  &  &2  \\ 
				Baza podataka				& 2 &  &  &  &  &  &   \\ 
				Dijagram razreda 			&  &  &  &  &  &  &   \\ 
				Dijagram stanja				&  &  &  &  &  &  &  \\ 
				Dijagram aktivnosti 		& 1 &  &  &  &  &  &  \\ 
				Dijagram komponenti			& 1 &  &  &  &  &  &  \\ 
				Korištene tehnologije i alati 		& 2 &  &  &  &  &  &  \\ 
				Ispitivanje programskog rješenja 	&  &  &  &  &  &  &  \\ 
				Dijagram razmještaja			& 1 &  &  &  &  &  &  \\ 
				Upute za puštanje u pogon 		&  &  &  &  &  &  &  \\  
				Dnevnik sastajanja 			& 2 &  &  &  &  &  &  \\ 
				Zaključak i budući rad 		&  &  &  &  &  &  &  \\  
				Popis literature 			&  &  &  &  &  & & 1  \\  
				\textit{Dizajniranje korisničkog sučelja} 			&  &  &  &  &  & 5 & 7  \\ 
				\textit{npr. izrada početne stranice} 				&  &  &  &  & 3 &  &  \\  
				\textit{izrada baze podataka} 		 			& 1 &  &  &  &  &  & \\  
				\textit{spajanje s bazom podataka} 							& 1 &  &  &  &  &  &  \\ 
				\textit{back end} 							&  1  &  &  &  &  &  \\  
				\textit{Formatiranje i lektura dokumentacije}			&  &  &  &  &  &  & 4\\ 
			\end{longtblr}
					
					
		\eject
		\section*{Dijagrami pregleda promjena}
		
		\textbf{\textit{dio 2. revizije}}\\
		
		\textit{Prenijeti dijagram pregleda promjena nad datotekama projekta. Potrebno je na kraju projekta generirane grafove s gitlaba prenijeti u ovo poglavlje dokumentacije. Dijagrami za vlastiti projekt se mogu preuzeti s gitlab.com stranice, u izborniku Repository, pritiskom na stavku Contributors.}
		
	
