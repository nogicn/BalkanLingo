\chapter{Specifikacija programske potpore}
		
	\section{Funkcionalni zahtjevi}
			
			\textbf{\textit{dio 1. revizije}}\\
			
			\textit{Navesti \textbf{dionike} koji imaju \textbf{interes u ovom sustavu} ili  \textbf{su nositelji odgovornosti}. To su prije svega korisnici, ali i administratori sustava, naručitelji, razvojni tim.}\\
				
			\textit{Navesti \textbf{aktore} koji izravno \textbf{koriste} ili \textbf{komuniciraju sa sustavom}. Oni mogu imati inicijatorsku ulogu, tj. započinju određene procese u sustavu ili samo sudioničku ulogu, tj. obavljaju određeni posao. Za svakog aktora navesti funkcionalne zahtjeve koji se na njega odnose.}\\
			
			
			\noindent \textbf{Dionici:}
			
			\begin{packed_enum}
				
				\item Dionik 1
				\item Dionik 2				
				\item ...
				
			\end{packed_enum}
			
			\noindent \textbf{Aktori i njihovi funkcionalni zahtjevi:}
			
			
			\begin{packed_enum}
				\item  \underbar{Aktor 1 (inicijator) može:}
				
				\begin{packed_enum}
					
					\item funkcionalnost 1
					\item funkcionalnost 2
					\begin{packed_enum}
						
						\item  podfunkcionalnost 1 
						\item  podfunkcionalnost 2
				
					\end{packed_enum}
					\item  funkcionalnost 3
					
				\end{packed_enum}
			
				\item  \underbar{Aktor 2 (sudionik) može:}
				
				\begin{packed_enum}
					
					\item funkcionalnost 1
					\item funkcionalnost 2
					
				\end{packed_enum}
			\end{packed_enum}
			
			\eject 
			
			
				
			\subsection{Obrasci uporabe}
				
				\textbf{\textit{dio 1. revizije}}
				
				\subsubsection{Opis obrazaca uporabe}
					\textit{Funkcionalne zahtjeve razraditi u obliku obrazaca uporabe. Svaki obrazac je potrebno razraditi prema donjem predlošku. Ukoliko u nekom koraku može doći do odstupanja, potrebno je to odstupanje opisati i po mogućnosti ponuditi rješenje kojim bi se tijek obrasca vratio na osnovni tijek.}\\
					

					\noindent \underbar{\textbf{UC1 Pristup stranici}}
					\begin{packed_item}
						\item \textbf{Glavni sudionik:} Korisnik
						\item \textbf{Cilj:} Dohvaćanje početne stranice
						\item \textbf{Sudionici:} Server
						\item \textbf{Preduvjet:} -
						\item \textbf{Opis osnovnog tijeka:}
						\begin{packed_enum}
							\item Korisnik upisuje URL
							\item Server dohvaća početnu stranicu
						\end{packed_enum}
						\item \textbf{Opis mogućih odstupanja:}
						\begin{packed_item}
							\item[2.a] Server nije u funkciji
						\end{packed_item}
					\end{packed_item}

					\noindent \underbar{\textbf{UC2 Prijava u sustav}}
					\begin{packed_item}
						\item \textbf{Glavni sudionik:} Korisnik
						\item \textbf{Cilj:} Prijava u korisnički račun
						\item \textbf{Sudionici:} Baza podataka
						\item \textbf{Preduvjet:} Postoji korisnički račun u bazi
						\item \textbf{Opis osnovnog tijeka:}
						\begin{packed_enum}
							\item Odabrana login opcija
							\item Upis email adrese
							\item Unos ispravne lozinke
							\item Pritisnuti Login dugme
							\item Server dohvaća podatke o korisniku
						\end{packed_enum}
						\item \textbf{Opis mogućih odstupanja:}
						\begin{packed_item}
							\item[2.a] Neispravna ili nepostojeća email adresa
							\item[3.a] Neispravna lozinka
							\item[5.a] Nešto sa serverom ili bazom????
						\end{packed_item}
					\end{packed_item}
					\noindent \underbar{\textbf{UC3 Registracija u sustav}}
					\begin{packed_item}
						\item \textbf{Glavni sudionik:} Korisnik
						\item \textbf{Cilj:} Registracija novog korisničkog računa
						\item \textbf{Sudionici:} Baza podataka
						\item \textbf{Preduvjet:} Korisnik nema račun
						\item \textbf{Opis osnovnog tijeka:}
						\begin{packed_enum}
							\item Odabrana opcija za registraciju
							\item Unos imena i prezimena
							\item Unos email adrese
							\item Pritisnuto dugme za Registraciju
						\end{packed_enum}
						\item \textbf{Opis mogućih odstupanja:}
						\begin{packed_item}
							\item[3.a] Već postoji račun s unesenom email adresom
							\item[5.a] Nije uspjelo generiranje jednokratne šifre?
						\end{packed_item}
					\end{packed_item}

					\noindent \underbar{\textbf{UC4.a Korisnik dodaje rječnik za učenje (Dashboard)}}
					\begin{packed_item}
						\item \textbf{Glavni sudionik:} Korisnik
						\item \textbf{Cilj:} Prikaz i odabir dostupnih rječnika
						\item \textbf{Sudionici:} Baza podataka
						\item \textbf{Preduvjet:} Korisnik je prijavljen
						\item \textbf{Opis osnovnog tijeka:}
						\begin{packed_enum}
							\item Pritisnuto dugme + (Opcija za odabir rječnika)
							\item Prikaz i odabir dostupnih rječnika
						\end{packed_enum}
						\item \textbf{Opis mogućih odstupanja:}
						\begin{packed_item}
							\item[2.a] Nema dostupnih rječnika
							\item[2.b] Pokušaj dodavanja već dodanog rječnika
						\end{packed_item}
					\end{packed_item}

					\noindent \underbar{\textbf{UC4.b Korisnik briše/odustaje od učenja jezika}}
					\begin{packed_item}
						\item \textbf{Glavni sudionik:} Korisnik
						\item \textbf{Cilj:} Brisanje rječnika s korisničkog računa
						\item \textbf{Sudionici:} Baza podataka
						\item \textbf{Preduvjet:} Korisnik je prijavljen, korisnik ima barem jedan rječnik dodan
						\item \textbf{Opis osnovnog tijeka:}
						\begin{packed_enum}
							\item Korisnik odabere rječnik
							\item Korisnik pritisne dugme za brisanje
						\end{packed_enum}
						\item \textbf{Opis mogućih odstupanja:}
					\end{packed_item}

					\noindent \underbar{\textbf{UC5.a Administrator dodaje rječnik}}
					\begin{packed_item}
						\item \textbf{Glavni sudionik:} Administrator
						\item \textbf{Cilj:} Dodavanje rječnika u sustav
						\item \textbf{Sudionici:} Baza podataka
						\item \textbf{Preduvjet:} Prijava s administratorskim privilegijama
						\item \textbf{Opis osnovnog tijeka:}
						\begin{packed_enum}
							\item Pritisnuto dugme +
							\item Otvaranje forme za dodavanje rječnika
							\item Unos imena rječnika
							\item Unos jezika rječnika
							\item Pritisnuto dugme Dodaj rječnik
						\end{packed_enum}
						\item \textbf{Opis mogućih odstupanja:}
						\begin{packed_item}
							\item[5.a] Dodavanje već postojećeg rječnika
							\item[5.b] Neuspjelo dodavanje rječnika u bazu podataka
						\end{packed_item}
					\end{packed_item}

					\noindent \underbar{\textbf{UC5.b Administrator briše rječnik}}
					\begin{packed_item}
						\item \textbf{Glavni sudionik:} Administrator
						\item \textbf{Cilj:} Brisanje rječnika iz sustava
						\item \textbf{Sudionici:} Baza podataka
						\item \textbf{Preduvjet:} Prijava s administratorskim privilegijama, postoji rječnik u sustavu
						\item \textbf{Opis osnovnog tijeka:}
						\begin{packed_enum}
							\item Administrator hovera preko rječnika
							\item Administrator pritisne ikonu kanti za smeće koja se pojavila na rječniku
						\end{packed_enum}
						\item \textbf{Opis mogućih odstupanja:}
						\begin{packed_item}
							\item 1. Nema rječnika u sustavu
						\end{packed_item}
					\end{packed_item}

					\noindent \underbar{\textbf{UC5.c Administrator dodaje riječi}}
					\begin{packed_item}
						\item \textbf{Glavni sudionik:} Administrator
						\item \textbf{Cilj:} Uređivanje rječnika
						\item \textbf{Sudionici:} Baza podataka
						\item \textbf{Preduvjet:} Prijava s administratorskim privilegijama, postoji rječnik
						\item \textbf{Opis osnovnog tijeka:}
						\begin{packed_enum}
							\item Kod rječnika se stisne gumb za uređivanje
							\item Otvori se forma za pretraživanje riječi tog jezika
							\item Nakon pronalaska tražene riječi stisni gumb + pored nje
						\end{packed_enum}
						\item \textbf{Opis mogućih odstupanja:}
						\begin{packed_item}
							\item 1. Nema napravljenog rječnika
							\item 2. Problem s API-jem za dohvaćanje riječi iz udaljenog rječnika
							\item 3. Riječ već u rječniku, opcija za uklanjanje iz rječnika
						\end{packed_item}
					\end{packed_item}

					\noindent \underbar{\textbf{UC5.d Administrator briše riječi}}
					\begin{packed_item}
						\item \textbf{Glavni sudionik:} Administrator
						\item \textbf{Cilj:} Uređivanje rječnika
						\item \textbf{Sudionici:} Baza podataka
						\item \textbf{Preduvjet:} Prijava s administratorskim privilegijama, postoji rječnik
						\item \textbf{Opis osnovnog tijeka:}
						\begin{packed_enum}
							\item Kod rječnika se stisne gumb za uređivanje
							\item Otvori se forma za pretraživanje riječi tog jezika
							\item Nakon pronalaska tražene riječi stisni gumb - pored nje
						\end{packed_enum}
						\item \textbf{Opis mogućih odstupanja:}
						\begin{packed_item}
							\item 1. Nema napravljenog rječnika
							\item 2. Problem s API-jem za dohvaćanje riječi iz udaljenog rječnika
							\item 3. Riječ nije u rječniku, opcija za dodavanje iz rječnika
						\end{packed_item}
					\end{packed_item}
					\noindent \underbar{\textbf{UC6.a Admin daje korisniku administratorske privilegije}}
					\begin{packed_item}
						\item \textbf{Glavni sudionik:} Administrator
						\item \textbf{Cilj:} Dodjeljivanje administratorskih privilegija korisniku
						\item \textbf{Sudionici:} Baza podataka
						\item \textbf{Preduvjet:} Ima administratorske privilegije
						\item \textbf{Opis osnovnog tijeka:}
						\begin{packed_enum}
							\item Otvara admin dashboard
							\item Upisuje email korisnika
							\item Pronađen korisnik
							\item Pritiska opciju za dodavanje admin privilegija
							\item Korisniku se dodaju administratorske privilegije
						\end{packed_enum}
						\item \textbf{Opis mogućih odstupanja:}
						\begin{packed_item}
							\item 3.a Korisnik ne postoji
							\item 5.a Neuspjelo dodjeljivanje administratorskih privilegija
						\end{packed_item}
					\end{packed_item}

					\noindent \underbar{\textbf{UC6.b Admin oduzima korisniku administratorske privilegije}}
					\begin{packed_item}
						\item \textbf{Glavni sudionik:} Administrator
						\item \textbf{Cilj:} Oduzimanje administratorskih privilegija korisniku
						\item \textbf{Sudionici:} Baza podataka
						\item \textbf{Preduvjet:} Ima administratorske privilegije
						\item \textbf{Opis osnovnog tijeka:}
						\begin{packed_enum}
							\item Otvara admin dashboard
							\item Upisuje email korisnika
							\item Pronađen korisnik
							\item Pritiska opciju za micanje admin privilegija
							\item Korisniku se oduzimaju administratorske privilegije
						\end{packed_enum}
						\item \textbf{Opis mogućih odstupanja:}
						\begin{packed_item}
							\item 3.a Korisnik ne postoji
							\item 5.a Neuspjelo oduzimanje administratorskih privilegija
						\end{packed_item}
					\end{packed_item}

					\noindent \underbar{\textbf{UC7 Admin briše korisnički račun}}
					\begin{packed_item}
						\item \textbf{Glavni sudionik:} Administrator
						\item \textbf{Cilj:} Brisanje korisnika iz baze podataka
						\item \textbf{Sudionici:} Baza podataka
						\item \textbf{Preduvjet:} Ima administratorske privilegije
						\item \textbf{Opis osnovnog tijeka:}
						\begin{packed_enum}
							\item Otvara admin dashboard
							\item Upisuje email korisnika
							\item Pronađen korisnik
							\item Pritiska opciju za brisanje korisnika
							\item Korisnik se miče iz baze podataka
						\end{packed_enum}
						\item \textbf{Opis mogućih odstupanja:}
						\begin{packed_item}
							\item 3.a Korisnik ne postoji u bazi podataka
							\item 5.a Neuspjelo micanje korisnika iz baze podataka
						\end{packed_item}
					\end{packed_item}

					\noindent \underbar{\textbf{UC8.a Korisnik mijenja svoj email}}
					\begin{packed_item}
						\item \textbf{Glavni sudionik:} Korisnik
						\item \textbf{Cilj:} Izmjena email adrese
						\item \textbf{Sudionici:} Baza podataka
						\item \textbf{Preduvjet:} Korisnik je već prijavljen u sustav
						\item \textbf{Opis osnovnog tijeka:}
						\begin{packed_enum}
							\item Korisnik odabire opciju promjene email adrese
							\item Korisnik unosi trenutnu lozinku
							\item Korisnik unosi novu email adresu
							\item Korisnik unosi lozinku ponovno
							\item Korisnikova email adresa postaje novonavedena adresa
							\item Poruka o uspješnoj promjeni email adrese
						\end{packed_enum}
						\item \textbf{Opis mogućih odstupanja:}
						\begin{packed_item}
							\item 1.a Korisnik nije prijavljen u sustav
							\item 1.b Preusmjeravanje na stranicu za prijavu
							\item 2.a Korisnik je krivo unio svoju lozinku prvi put
							\item 2.b Ponovni unos lozinke
							\item 3.a Korisnik je unio neispravnu (nevažeću) email adresu
							\item 4.a Korisnik je krivo unio svoju lozinku drugi put
							\item 5.a Nova email adresa se nije unijela u bazu podataka
							\item 5.b Poruka o neuspješnoj promjeni email adrese
						\end{packed_item}
					\end{packed_item}

					\noindent \underbar{\textbf{UC8.b Korisnik mijenja svoju lozinku}}
					\begin{packed_item}
						\item \textbf{Glavni sudionik:} Korisnik
						\item \textbf{Cilj:} Izmjena korisničke lozinke
						\item \textbf{Sudionici:} Baza podataka
						\item \textbf{Preduvjet:} Korisnik ima korisnički račun
						\item \textbf{Opis osnovnog tijeka:}
						\begin{packed_enum}
							\item Korisnik odabire opciju promjene lozinke
							\item Korisnik unosi trenutnu lozinku
							\item Korisnik unosi novu lozinku
							\item Korisnik unosi novu lozinku ponovno
							\item Lozinka se mijenja
							\item Poruka o uspješnoj promjeni lozinke
						\end{packed_enum}
						\item \textbf{Opis mogućih odstupanja:}
						\begin{packed_item}
							\item 1.a Korisnik nema korisnički račun
							\item 1.b Preusmjeravanje na stranicu za registraciju
							\item 2.a Korisnik je krivo unio svoju trenutnu lozinku
							\item 3.a Unesena nova lozinka nije u skladu s pravilima za sigurnost lozinke
							\item 4.a Korisnik je krivo unio novu lozinku ponovno
							\item 5.a Lozinka nije promijenjena u bazi podataka
							\item 5.b Poruka o neuspješnoj promjeni lozinke
						\end{packed_item}
					\end{packed_item}
					\noindent \underbar{\textbf{UC9 Korisnik se odjavljuje sa svog računa}}
					\begin{packed_item}
						\item \textbf{Glavni sudionik:} Korisnik
						\item \textbf{Cilj:} Odjava sa web-aplikacije
						\item \textbf{Sudionici:} Baza podataka
						\item \textbf{Preduvjet:} Korisnik je trenutno ulogiran u aplikaciji
						\item \textbf{Opis osnovnog tijeka:}
						\begin{packed_enum}
							\item Na vrhu stranice pritisne gumb za odjavu iz aplikacije
							\item Server odjavljuje korisnika
						\end{packed_enum}
						\item \textbf{Opis mogućih odstupanja:}
						\begin{packed_item}
							\item 2.a Nešto sa serverom ili bazom????
							\item 2.b Račun je obrisan dok session traje
						\end{packed_item}
					\end{packed_item}

					\noindent \underbar{\textbf{UC10 Korisnik bira učenje odabranog rječnika}}
					\begin{packed_item}
						\item \textbf{Glavni sudionik:} Korisnik
						\item \textbf{Cilj:} Odabir učenja odabranog rječnika
						\item \textbf{Sudionici:} Baza podataka
						\item \textbf{Preduvjet:} Korisnik je trenutno ulogiran u aplikaciji i ima dodan rječnik za svoj način učenja
						\item \textbf{Opis osnovnog tijeka:}
						\begin{packed_enum}
							\item Korisnik na glavoj stranici odabire rječnik koji želi učiti
						\end{packed_enum}
						\item \textbf{Opis mogućih odstupanja:}
						\begin{packed_item}
							\item 1.a Nešto sa serverom ili bazom????
						\end{packed_item}
					\end{packed_item}

					\noindent \underbar{\textbf{UC11.a Korisnik uči riječi modom upit (engleske) riječi uz odabir (hrvatskog) prijevoda}}
					\begin{packed_item}
						\item \textbf{Glavni sudionik:} Korisnik
						\item \textbf{Cilj:} Učenje riječi modom upit (engleske) riječi uz odabir (hrvatskog) prijevoda
						\item \textbf{Sudionici:} Baza podataka
						\item \textbf{Preduvjet:} Korisnik je odabrao rječnik za učenje engleskog jezika
						\item \textbf{Opis osnovnog tijeka:}
						\begin{packed_enum}
							\item Korisnik dobiva riječ za učenje modom upit (engleske) riječi uz odabir (hrvatskog) prijevoda
							\item Korisnik odabire neku hrvatsku riječ
							\item Ovisno o točnosti odabira, dobiva povratnu informaciju
							\item Izlaz iz trenutnog moda učenja
						\end{packed_enum}
						\item \textbf{Opis mogućih odstupanja:}
						\begin{packed_item}
							\item 1.a Nešto sa serverom ili bazom????
							\item 3.a Ne dobije povratnu informaciju
							\item 4.a Aplikacija ne izade iz moda
						\end{packed_item}
					\end{packed_item}

					\noindent \underbar{\textbf{UC11.b Korisnik uči riječi modom upit (hrvatske) riječi uz odabir (engleskog) prijevoda}}
					\begin{packed_item}
						\item \textbf{Glavni sudionik:} Korisnik
						\item \textbf{Cilj:} Učenje riječi modom upit (hrvatske) riječi uz odabir (engleskog) prijevoda
						\item \textbf{Sudionici:} Baza podataka
						\item \textbf{Preduvjet:} Korisnik je odabrao rječnik za učenje hrvatskog jezika
						\item \textbf{Opis osnovnog tijeka:}
						\begin{packed_enum}
							\item Korisnik dobiva riječ za učenje modom upit (hrvatske) riječi uz odabir (engleskog) prijevoda
							\item Korisnik odabire neku hrvatsku riječ
							\item Ovisno o točnosti odabira, dobiva povratnu informaciju
							\item Izlaz iz trenutnog moda učenja
						\end{packed_enum}
						\item \textbf{Opis mogućih odstupanja:}
						\begin{packed_item}
							\item 1.a Nešto sa serverom ili bazom????
							\item 3.a Ne dobije povratnu informaciju
							\item 4.a Aplikacija ne izade iz moda
						\end{packed_item}
					\end{packed_item}

					\noindent \underbar{\textbf{UC11.c Korisnik uči riječi modom upit izgovorom (engleske) riječi uz pisanje riječi na (engleskom)}}
					\begin{packed_item}
						\item \textbf{Glavni sudionik:} Korisnik
						\item \textbf{Cilj:} Učenje riječi modom upit izgovorom (engleske) riječi uz pisanje riječi na (engleskom)
						\item \textbf{Sudionici:} Baza podataka
						\item \textbf{Preduvjet:} Korisnik je odabrao rječnik za učenje engleskog jezika
						\item \textbf{Opis osnovnog tijeka:}
						\begin{packed_enum}
							\item Korisnik dobiva zvučni zapis koji može slušati
							\item Korisnik upisuje riječ u tekstualnu kutiju
							\item Korisnik stisne gumb za slanje riječi
							\item Ovisno o točnosti upisa, dobiva povratnu informaciju
							\item Izlaz iz trenutnog moda učenja
						\end{packed_enum}
						\item \textbf{Opis mogućih odstupanja:}
						\begin{packed_item}
							\item 1.a Nešto sa serverom ili bazom????
							\item 1.b Zvučni zapis ne može se pustiti
							\item 3.a Ne radi gumb za slanje
							\item 4.a Aplikacija ne izade iz moda
						\end{packed_item}
					\end{packed_item}

					\noindent \underbar{\textbf{UC11.d Korisnik uči riječi modom upit tekstualnim oblikom (engleske) riječi uz snimanje izgovora}}
					\begin{packed_item}
						\item \textbf{Glavni sudionik:} Korisnik
						\item \textbf{Cilj:} Učenje riječi modom upit (engleske) riječi uz snimanje izgovora
						\item \textbf{Sudionici:} Baza podataka
						\item \textbf{Preduvjet:} Korisnik je odabrao rječnik za učenje engleskog jezika
						\item \textbf{Opis osnovnog tijeka:}
						\begin{packed_enum}
							\item Korisnik dobiva riječ za učenje modom upit tekstualnim oblikom (engleske) riječi uz snimanje izgovora
							\item Korisnik šalje glasovni zapis svog izgovora
							\item Ovisno o točnosti odabira, dobiva povratnu informaciju
							\item Izlaz iz trenutnog moda učenja
						\end{packed_enum}
						\item \textbf{Opis mogućih odstupanja:}
						\begin{packed_item}
							\item 1.a Nešto sa serverom ili bazom????
							\item 3.a Ne dobije povratnu informaciju
							\item 4.a Aplikacija ne izade iz moda
						\end{packed_item}
					\end{packed_item}
					\noindent \underbar{\textbf{UC9 Korisnik se odjavljuje sa svog računa}}
					\begin{packed_item}
						\item \textbf{Glavni sudionik:} Korisnik
						\item \textbf{Cilj:} Odjava sa web-aplikacije
						\item \textbf{Sudionici:} Baza podataka
						\item \textbf{Preduvjet:} Korisnik je trenutno ulogiran u aplikaciji
						\item \textbf{Opis osnovnog tijeka:}
						\begin{packed_enum}
							\item Na vrhu stranice pritisne gumb za odjavu iz aplikacije
							\item Server odjavljuje korisnika
						\end{packed_enum}
						\item \textbf{Opis mogućih odstupanja:}
						\begin{packed_item}
							\item 2.a Nešto sa serverom ili bazom????
							\item 2.b Račun je obrisan dok session traje
						\end{packed_item}
					\end{packed_item}

					\noindent \underbar{\textbf{UC10 Korisnik bira učenje odabranog rječnika}}
					\begin{packed_item}
						\item \textbf{Glavni sudionik:} Korisnik
						\item \textbf{Cilj:} Odabir učenja odabranog rječnika
						\item \textbf{Sudionici:} Baza podataka
						\item \textbf{Preduvjet:} Korisnik je trenutno ulogiran u aplikaciji i ima dodan rječnik za svoj način učenja
						\item \textbf{Opis osnovnog tijeka:}
						\begin{packed_enum}
							\item Korisnik na glavoj stranici odabire rječnik koji želi učiti
						\end{packed_enum}
						\item \textbf{Opis mogućih odstupanja:}
						\begin{packed_item}
							\item 1.a Nešto sa serverom ili bazom????
						\end{packed_item}
					\end{packed_item}

					\noindent \underbar{\textbf{UC11.a Korisnik uči riječi modom upit (engleske) riječi uz odabir (hrvatskog) prijevoda}}
					\begin{packed_item}
						\item \textbf{Glavni sudionik:} Korisnik
						\item \textbf{Cilj:} Učenje riječi modom upit (engleske) riječi uz odabir (hrvatskog) prijevoda
						\item \textbf{Sudionici:} Baza podataka
						\item \textbf{Preduvjet:} Korisnik je odabrao rječnik za učenje engleskog jezika
						\item \textbf{Opis osnovnog tijeka:}
						\begin{packed_enum}
							\item Korisnik dobiva riječ za učenje modom upit (engleske) riječi uz odabir (hrvatskog) prijevoda
							\item Korisnik odabire neku hrvatsku riječ
							\item Ovisno o točnosti odabira, dobiva povratnu informaciju
							\item Izlaz iz trenutnog moda učenja
						\end{packed_enum}
						\item \textbf{Opis mogućih odstupanja:}
						\begin{packed_item}
							\item 1.a Nešto sa serverom ili bazom????
							\item 3.a Ne dobije povratnu informaciju
							\item 4.a Aplikacija ne izade iz moda
						\end{packed_item}
					\end{packed_item}

					\noindent \underbar{\textbf{UC11.b Korisnik uči riječi modom upit (hrvatske) riječi uz odabir (engleskog) prijevoda}}
					\begin{packed_item}
						\item \textbf{Glavni sudionik:} Korisnik
						\item \textbf{Cilj:} Učenje riječi modom upit (hrvatske) riječi uz odabir (engleskog) prijevoda
						\item \textbf{Sudionici:} Baza podataka
						\item \textbf{Preduvjet:} Korisnik je odabrao rječnik za učenje hrvatskog jezika
						\item \textbf{Opis osnovnog tijeka:}
						\begin{packed_enum}
							\item Korisnik dobiva riječ za učenje modom upit (hrvatske) riječi uz odabir (engleskog) prijevoda
							\item Korisnik odabire neku hrvatsku riječ
							\item Ovisno o točnosti odabira, dobiva povratnu informaciju
							\item Izlaz iz trenutnog moda učenja
						\end{packed_enum}
						\item \textbf{Opis mogućih odstupanja:}
						\begin{packed_item}
							\item 1.a Nešto sa serverom ili bazom????
							\item 3.a Ne dobije povratnu informaciju
							\item 4.a Aplikacija ne izade iz moda
						\end{packed_item}
					\end{packed_item}

					\noindent \underbar{\textbf{UC11.c Korisnik uči riječi modom upit izgovorom (engleske) riječi uz pisanje riječi na (engleskom)}}
					\begin{packed_item}
						\item \textbf{Glavni sudionik:} Korisnik
						\item \textbf{Cilj:} Učenje riječi modom upit izgovorom (engleske) riječi uz pisanje riječi na (engleskom)
						\item \textbf{Sudionici:} Baza podataka
						\item \textbf{Preduvjet:} Korisnik je odabrao rječnik za učenje engleskog jezika
						\item \textbf{Opis osnovnog tijeka:}
						\begin{packed_enum}
							\item Korisnik dobiva zvučni zapis koji može slušati
							\item Korisnik upisuje riječ u tekstualnu kutiju
							\item Korisnik stisne gumb za slanje riječi
							\item Ovisno o točnosti upisa, dobiva povratnu informaciju
							\item Izlaz iz trenutnog moda učenja
						\end{packed_enum}
						\item \textbf{Opis mogućih odstupanja:}
						\begin{packed_item}
							\item 1.a Nešto sa serverom ili bazom????
							\item 1.b Zvučni zapis ne može se pustiti
							\item 3.a Ne radi gumb za slanje
							\item 4.a Aplikacija ne izade iz moda
						\end{packed_item}
					\end{packed_item}

					\noindent \underbar{\textbf{UC11.d Korisnik uči riječi modom upit tekstualnim oblikom (engleske) riječi uz snimanje izgovora}}
					\begin{packed_item}
						\item \textbf{Glavni sudionik:} Korisnik
						\item \textbf{Cilj:} Učenje riječi modom upit (engleske) riječi uz snimanje izgovora
						\item \textbf{Sudionici:} Baza podataka
						\item \textbf{Preduvjet:} Korisnik je odabrao rječnik za učenje engleskog jezika
						\item \textbf{Opis osnovnog tijeka:}
						\begin{packed_enum}
							\item Korisnik dobiva riječ za učenje modom upit tekstualnim oblikom (engleske) riječi uz snimanje izgovora
							\item Korisnik šalje glasovni zapis svog izgovora
							\item Ovisno o točnosti odabira, dobiva povratnu informaciju
							\item Izlaz iz trenutnog moda učenja
						\end{packed_enum}
						\item \textbf{Opis mogućih odstupanja:}
						\begin{packed_item}
							\item 1.a Nešto sa serverom ili bazom????
							\item 3.a Ne dobije povratnu informaciju
							\item 4.a Aplikacija ne izade iz moda
						\end{packed_item}
					\end{packed_item}

					\noindent \underbar{\textbf{UC12 Dolazak na postavke korisničkog računa}}
					\begin{packed_item}
						\item \textbf{Glavni sudionik:} Korisnik
						\item \textbf{Cilj:} Korisnik dolazi na stranicu za postavke korisničkog računa
						\item \textbf{Sudionici:} Baza podataka
						\item \textbf{Preduvjet:} Korisnik je prijavljen na račun
						\item \textbf{Opis osnovnog tijeka:}
						\begin{packed_enum}
							\item Korisnik stisne na ikonu svoga profila
						\end{packed_enum}
						\item \textbf{Opis mogućih odstupanja:}
						\begin{packed_item}
							\item 1.a Nešto sa serverom ili bazom????
						\end{packed_item}
					\end{packed_item}


				\subsubsection{Dijagrami obrazaca uporabe}
					
					\textit{Prikazati odnos aktora i obrazaca uporabe odgovarajućim UML dijagramom. Nije nužno nacrtati sve na jednom dijagramu. Modelirati po razinama apstrakcije i skupovima srodnih funkcionalnosti.}
				\eject		
				
			\subsection{Sekvencijski dijagrami}
				
				\textbf{\textit{dio 1. revizije}}\\
				
				\textit{Nacrtati sekvencijske dijagrame koji modeliraju najvažnije dijelove sustava (max. 4 dijagrama). Ukoliko postoji nedoumica oko odabira, razjasniti s asistentom. Uz svaki dijagram napisati detaljni opis dijagrama.}
				\eject
	
		\section{Ostali zahtjevi}
		
			\textbf{\textit{dio 1. revizije}}\\
		 
			 \textit{Nefunkcionalni zahtjevi i zahtjevi domene primjene dopunjuju funkcionalne zahtjeve. Oni opisuju \textbf{kako se sustav treba ponašati} i koja \textbf{ograničenja} treba poštivati (performanse, korisničko iskustvo, pouzdanost, standardi kvalitete, sigurnost...). Primjeri takvih zahtjeva u Vašem projektu mogu biti: podržani jezici korisničkog sučelja, vrijeme odziva, najveći mogući podržani broj korisnika, podržane web/mobilne platforme, razina zaštite (protokoli komunikacije, kriptiranje...)... Svaki takav zahtjev potrebno je navesti u jednoj ili dvije rečenice.}
			 
			 
			 
	