\chapter{Implementacija i korisničko sučelje}
		
		
		\section{Korištene tehnologije i alati}
			
		\subsection*{Komunikacija}
			Tijekom razvoja projekta koristili smo mnoge tehnologije kako bi olakšali i ubrzali komunikaciju. Većina interne projektne komunikacije odvijala se unutar privatnog servera u aplikaciji \textbf{\href{https://discord.com/}{Discord}}, dok je ostatak komunikacije bio preko \textbf{\href{https://www.microsoft.com/hr-hr/microsoft-teams/download-app}{MS Teams-a}}.
			Discord je aplikacija koja omogućava brzo postavljanje kanala za razgovor i brzo odvajanje logičkih jedinica projekta (front-end, back-end, dokumentacija).
			Za komunikaciju o projektu, zadatcima i projektnim obavijestima s asistenticom koristili smo \href{https://www.microsoft.com/hr-hr/microsoft-teams/download-app}{MS Teams}.


		\subsection*{Dokumentacija i verziranje aplikacije}
			Dokumentacija je napisana u sustavu \textbf{\href{https://www.latex-project.org/}{LaTeX}}. {LaTeX} je jezik koji omogućuje lagano pisanje dokumentacije kao i mogućnost brze iteracije i izmjena podataka. Za izradu grafova u projektu koristili smo \textbf{\href{https://www.visual-paradigm.com/}{Visual Paradigm}} i \textbf{\href{https://www.lucidchart.com/pages/}{Lucidchart}}. Verziranje aplikacije je bitan dio u procesu razvijanja aplikacije jer omogućava nadgledanje koji dio koda je kada dodan i mogućnost vraćanja u prošlost ako neki dio koda ne radi u budućnosti. Za verziranje koda koristili smo \textbf{\href{https://git-scm.com/}{Git}}, a za spremanje tih promjena koristili smo udaljeni repozitorij na \textbf{\href{https://github.com/}{GitHub-u}}.

		\subsection*{Aplikacija}
			Aplikacija je napisana u \textbf{\href{https://developer.mozilla.org/en-US/docs/Web/JavaScript}{JavaScript-u}} koristeći \textbf{\href{https://nodejs.org/en}{Node.js}} i \textbf{\href{https://www.npmjs.com/package/better-sqlite3}{SQLITE3}} bazu podataka. Node je okruženje za izvršavanje JavaScripta na bilo kojoj platformi izvan preglednika te se koristi kao backend server. Naša aplikacija nema odvojen front-end i back-end jer svu logiku izvršavamo na backendu u Node-u. Na backendu koristimo i \textbf{\href{https://expressjs.com/}{Express}} radni okvir koji nam olakšava pisanje logike HTTP zahtjeva. Za stvaranje HTML-a koji se onda šalje korisniku koristimo \href{https://ejs.co/}{ejs} šablonski jezik. Za SQLITE implementaciju koristimo \textbf{\href{https://www.npmjs.com/package/better-sqlite3}{better-sqlite3}} biblioteku koja nam omogućava brzo i jednostavno stvaranje baze podataka kao i bolje debugiranje problema oko upita.

		\subsection*{Razvojno okruženje}
			Za razvojno okruženje koristili smo \textbf{\href{https://code.visualstudio.com/}{Visual Studio Code}}. Visual Studio Code je besplatni izvorni kod editor koji je razvila tvrtka Microsoft. Za stiliziranje koda koristili smo \textbf{\href{https://prettier.io/}{Prettier}}. Prettier je alat koji nam omogućava da kod bude stilski konzistentan i da se ne moramo brinuti o stilu koda.

		\subsection*{Korisni linkovi}
			\begin{itemize}
				\item Discord: \url{https://discord.com/}
				\item Microsoft Teams: \url{https://www.microsoft.com/hr-hr/microsoft-teams/download-app}
				\item LaTeX: \url{https://www.latex-project.org/}
				\item Visual Paradigm: \url{https://www.visual-paradigm.com/}
				\item Lucidchart: \url{https://www.lucidchart.com/pages/}
				\item Git: \url{https://git-scm.com/}
				\item GitHub: \url{https://github.com/}
				\item JavaScript: \url{https://developer.mozilla.org/en-US/docs/Web/JavaScript}
				\item Node.js: \url{https://nodejs.org/en}
				\item Express: \url{https://expressjs.com/}
				\item EJS: \url{https://ejs.co/}
				\item Visual Studio Code: \url{https://code.visualstudio.com/}
				\item Prettier: \url{https://prettier.io/}
				\item Better SQLite3: \url{https://www.npmjs.com/package/better-sqlite3}
			\end{itemize}

				
			\eject 
		
	
		\section{Ispitivanje programskog rješenja}

			Testiranje koda ključan je korak u razvoju web aplikacije iz nekoliko razloga:
			\begin{itemize}
				\item Omogućava identifikaciju i ispravljanje grešaka prije nego što aplikacija dosegne korisnike.
				\item Osiguravamo da je aplikacija pouzdana, te radi dosljedno i bez neočekivanih ponašanja.
				\item Otkrivanjem i rješavanjem problema u ranim fazama razvoja, smanjujemo troškove održavanja i podrške nakon implementacije.
			\end{itemize}

			Za automatsko testiranje naše aplikacije koristili smo supertest.
			Supertest je biblioteka za testiranje HTTP/HTTPS za Node.js koja nam omogućava simulaciju HTTP zahtjeva i provjeru odgovora.
			Ovaj alat koristimo kako bismo automatizirali proces testiranja naše web aplikacije.
			
			\subsection{Ispitivanje komponenti}
			\textit{Potrebno je provesti ispitivanje jedinica (engl. unit testing) nad razredima koji implementiraju temeljne funkcionalnosti. Razraditi \textbf{minimalno 6 ispitnih slučajeva} u kojima će se ispitati redovni slučajevi, rubni uvjeti te izazivanje pogreške (engl. exception throwing). Poželjno je stvoriti i ispitni slučaj koji koristi funkcionalnosti koje nisu implementirane. Potrebno je priložiti izvorni kôd svih ispitnih slučajeva te prikaz rezultata izvođenja ispita u razvojnom okruženju (prolaz/pad ispita). }
			
			
			
			\subsection{Ispitivanje sustava}
			
			U cilju osiguranja sigurnosti, pravilnog funkcioniranja i pristupačnosti specifičnih sadržaja, 
			implementirali smo tri odvojena seta testova prilagođena različitim korisničkim ulogama: gostima, korisnicima i administratorima.
			U nastavku je prikazano par primjeraka iz svakog seta.

			\subsubsection{Testovi za goste}
			Testovi za goste provjeravaju da gosti mogu pristupiti svim javnim stranicama aplikacije, dok im je pristup ostalim stranicama onemogućen.

			Ispitni slučaj \textbf{Root path test}
			\begin{itemize}
				\item \textbf{Opis:} provjerava da gost može pristupiti početnoj stranici aplikacije.
				\item \textbf{Ulaz:} GET zahtjev za početnu stranicu.
				\item \textbf{Očekivani rezultat:} Statusni kod 200.
				\begin{figure}[h]
					\centering
					\includegraphics[width=0.8\textwidth]{slike/testovi/guest_root_test.png}
					\caption{Opis slike koja ilustrira ispitivanje sustava za goste.}
					\label{fig:testovi_gosti_slika}
				\end{figure}
			\end{itemize}

			Ispitni slučaj \textbf{Edit user test}
			\begin{itemize}
				\item \textbf{Opis:} provjerava da gost ne smije pristupiti stranici za uređivanje korisnika.
				\item \textbf{Ulaz:} GET zahtjev za /user/edit
				\item \textbf{Očekivani rezultat:} Statusni kod 403.
				\begin{figure}[h]
					\centering
					\includegraphics[width=0.8\textwidth]{slike/testovi/guest_edit_profile_test.png}
					\caption{Opis slike koja ilustrira ispitivanje sustava za goste.}
					\label{fig:testovi_gosti_slika}
				\end{figure}
			\end{itemize}
			
			\subsubsection{Testovi za korisnike}
			Testovi za korisnike provjeravaju da korisnici mogu pristupiti svim javnim stranicama aplikacije,
			kao i stranicama koje su namijenjene samo korisnicima,
			dok im je zabranjen pristup stranicama namijenjenim administratorima.

			Ispitni slučaj \textbf{Adding dictionary test}
			\begin{itemize}
				\item \textbf{Opis:} provjerava da koristnik može dodati novi rječnik.
				\item \textbf{Ulaz:} GET zahtjev za /dictionary/addDictionary
				\item \textbf{Očekivani rezultat:} Statusni kod 200.
				\begin{figure}[h]
					\centering
					\includegraphics[width=0.8\textwidth]{slike/testovi/user_add_dictionary_test.png}
					\caption{Opis slike koja ilustrira ispitivanje sustava za korisnika.}
					\label{fig:testovi_korisnik_slika}
				\end{figure}
			\end{itemize}

			Ispitni slučaj \textbf{Editing word test}
			\begin{itemize}
				\item \textbf{Opis:} provjerava da gost ne smije promijeniti detalje riječi u riječniku.
				\item \textbf{Ulaz:} GET zahtjev za /dictionary/editWord/1
				\item \textbf{Očekivani rezultat:} Statusni kod 403.
				\begin{figure}[h]
					\centering
					\includegraphics[width=0.8\textwidth]{slike/testovi/user_edit_word_test.png}
					\caption{Opis slike koja ilustrira ispitivanje sustava za korisnika.}
					\label{fig:testovi_korisnik_slika}
				\end{figure}
			\end{itemize}

			\subsubsection{Testovi za administratore}
			Testovi za administratore provjeravaju da administratori mogu pristupiti svim stranicama aplikacije.

			Ispitni slučaj \textbf{Editing word test}
			\begin{itemize}
				\item \textbf{Opis:} provjerava da administrator može promijeniti detalje riječi u riječniku.
				\item \textbf{Ulaz:} GET zahtjev za /dictionary/editWord/1
				\item \textbf{Očekivani rezultat:} Statusni kod 200.
				\begin{figure}[h]
					\centering
					\includegraphics[width=0.8\textwidth]{slike/testovi/admin_edit_word_test.png}
					\caption{Opis slike koja ilustrira ispitivanje sustava za admina.}
					\label{fig:testovi_admin_slika}
				\end{figure}
			\end{itemize}

			Ispitni slučaj \textbf{Admin adding dictionary test}
			\begin{itemize}
				\item \textbf{Opis:} provjerava da admin može stvoriti novi rječnik.
				\item \textbf{Ulaz:} GET zahtjev za /dictionary/addAdminDict
				\item \textbf{Očekivani rezultat:} Statusni kod 200.
				\begin{figure}[h]
					\centering
					\includegraphics[width=0.8\textwidth]{slike/testovi/admin_add_dictionary_test.png}
					\caption{Opis slike koja ilustrira ispitivanje sustava za admina.}
					\label{fig:testovi_admin_slika}
				\end{figure}
			\end{itemize}

			Ispitni slučaj \textbf{Dictionary root path test}
			\begin{itemize}
				\item \textbf{Opis:} provjerava da se ne može pristupiti korjenskom direktoriju rječnika.
				\item \textbf{Ulaz:} GET zahtjev za /dictionary
				\item \textbf{Očekivani rezultat:} Statusni kod 404.
				\begin{figure}[h]
					\centering
					\includegraphics[width=0.8\textwidth]{slike/testovi/admin_404_test.png}
					\caption{Opis slike koja ilustrira ispitivanje sustava za admina.}
					\label{fig:testovi_admin_slika}
				\end{figure}
			\end{itemize}
			
			\eject 
		
		
		\section{Dijagram razmještaja}
			
		Dijagram razmještaja opisuje topologiju sustava. Dijagram prikazuje sve potrebno za uspješnu komunikaciju sutava. Komunikacija kreće od korisničkog uređaja koji šalje HTTP zahtjev na poslužiteljsko računalo da vrati prikaz aplikacije. Node.js (Express) poslužitelj ovisno o zahtjevu može pogledati u sqlite bazu podataka kako bi dohvatio dodatne podatke poput liste rječnika. Nakon toga server stvara prikaz i šalje ga klijentu preko HTTP veze.
			\begin{figure}[h]
				\centering
				\includegraphics[width=1\textwidth]{dijagrami/DijagamRazmjestaja.jpg}
				\caption{Dijagram razmještaja}
				\label{fig:dijagram_razmjestaja}
			\end{figure}	

			\eject 
		
		\section{Upute za puštanje u pogon}
		
		\subsection*{Instaliranje nodejs-a i npm-a}
		\begin{enumerate}
			\item Preuzmite nodejs instalaciju s \textbf{\href{https://nodejs.org/en/download/}{službene stranice}}.
			\begin{figure}[h]
				\centering
				\includegraphics[width=0.8\textwidth]{slike/npm_install/0.png}
				\caption{Slika stranice za skidanje NodeJs-a}
				\label{fig:node_install_images}
			\end{figure}
			\item Pokrenite instalaciju.
			\begin{figure}[h]
				\centering
				\includegraphics[width=0.8\textwidth]{slike/npm_install/1.png}
				\caption{Prvi korak instalacije programa Node}
				\label{fig:node_install_images}
			\end{figure}
			\eject
			\item Odaberite direktorij za instalaciju
			\begin{figure}[h]
				\centering
				\includegraphics[width=0.6\textwidth]{slike/npm_install/2.png}
				\caption{Odabir direktorija instalacije programa Node}
				\label{fig:node_install_images}
			\end{figure}
			\item Kliknite Next
			\begin{figure}[h]
				\centering
				\includegraphics[width=0.6\textwidth]{slike/npm_install/3.png}
				\caption{Prvi korak instalacije programa Node}
				\label{fig:node_install_images}
			\end{figure}
			\item Provjerite je li nodejs uspješno instaliran:
			 \subitem node -v
			\item Provjerite je li npm uspješno instaliran:
			\subitem npm -v
		\end{enumerate}
		\subsection*{Ručno Instaliranje}
		Za ručno pokretanje aplikacije potrebna je Node.js verzija 20 ili novija.
		\begin{enumerate}
			\item Promijenite direktorij u mapu "app":
			\begin{verbatim}
				dir IzvorniKod/app 
			\end{verbatim}
			\item Instalirajte projektne module:
			\begin{verbatim}
				npm install
			\end{verbatim}
			\item Pokrenite aplikaciju:
			\begin{verbatim}
				npm run start
			\end{verbatim}
			Aplikacija se pokreće na portu 3000.
		\end{enumerate}
		\newpage
		\subsection*{Instaliranje Docker-a}
		\begin{enumerate}
			\item Preuzmite Docker instalaciju s \textbf{\href{https://docs.docker.com/get-docker/}{službene stranice}}.
			\begin{figure}[h]
				\centering
				\includegraphics[width=0.6\textwidth]{slike/docker_install/0.png}
				\caption{Skidanje Docker instalacije sa službene stranice}
			\end{figure}
			\newpage
			\item Pokrenite instalaciju i pritisnite OK kako biste nastavili
			\begin{figure}[h]
				\centering
				\includegraphics[width=0.6\textwidth]{slike/docker_install/2.png}
				
				\caption{Instalacija Docker aplikacije}
			\end{figure}
			\begin{figure}[h]
				\centering
				\includegraphics[width=0.6\textwidth]{slike/docker_install/3.png}
				\caption{Instalacija Docker aplikacije}
			\end{figure}
			\item Nakon instalacije morate resetirati operacijski sustav
			\begin{figure}[h]
				\centering
				\includegraphics[width=0.8\textwidth]{slike/docker_install/4.png}
				\caption{Skidanje Docker instalacije sa službene stranice}
			\end{figure}
			\newpage
			\item Nakon resetiranja sustava morate prihvatiti uvjete korištenja
			\begin{figure}[h]
				\centering
				\includegraphics[width=0.6\textwidth]{slike/docker_install/5.png}
				\caption{Instalacija Docker aplikacije}
			\end{figure}
			\item Kako bi koristili Docker na svom računalu morate instalirati WSL2 podsustav
			\begin{figure}[h]
				\centering
				\includegraphics[width=0.6\textwidth]{slike/docker_install/7.png}
				\includegraphics[width=0.6\textwidth]{slike/docker_install/6.png}
				\caption{Instalacija WSL2 podsustava}
			\end{figure}
			\newpage
			\item Pokrenite instalaciju
			\begin{figure}[h]
				\centering
				\includegraphics[width=0.6\textwidth]{slike/docker_install/8.png}
				\caption{Instalacija WSL2 podsustava}
			\end{figure}
			\item Kliknite Next 
			\begin{figure}[h]
				\centering
				\includegraphics[width=0.6\textwidth]{slike/docker_install/9.png}
				\caption{Instalacija WSL2 podsustava}
			\end{figure}
			\newpage
			\item Nakon instalacije WSL2 podsustava u Docker aplikaciji kliknite Restart
			\begin{figure}[h]
				\centering
				\includegraphics[width=0.6\textwidth]{slike/docker_install/10.png}
				\caption{Instalacija Docker aplikacije}
			\end{figure}
			\item Nakon resetiranja operacijskog sustava možete početi koristiti Docker
			\begin{figure}[h]
				\centering
				\includegraphics[width=0.6\textwidth]{slike/docker_install/11.png}
				\caption{Instalacija Docker aplikacije}
			\end{figure}
			\item Provjerite je li Docker uspješno instaliran:
			\begin{verbatim}
				docker -v
			\end{verbatim}
		\end{enumerate}

		\subsection*{Postavljanje Docker-a}
		Za pokretanje aplikacije pomoću Docker-a:
		\begin{enumerate}
			\item Promijenite direktorij u mapu "IzvorniKod":
			\begin{verbatim}
				dir IzvorniKod
			\end{verbatim}
			\item Izgradite Docker sliku pod nazivom "balkan-lingo":
			\begin{verbatim}
				docker build -t balkan-lingo .
			\end{verbatim}
			\item Pokrenite Docker kontejner i mapirajte port 3000 iz kontejnera na vaš poslužiteljski port:
			\begin{verbatim}
				docker run -p 3000:3000 balkan-lingo
			\end{verbatim}
			Za pokretanje kontejnera u pozadini koristite:
			\begin{verbatim}
				docker run -d -p 3000:3000 balkan-lingo
			\end{verbatim}
		\end{enumerate}
		
		\subsection*{Zaustavljanje Docker Kontejnera}
		Za zaustavljanje Docker kontejnera:
		\begin{enumerate}
			\item Navedite sve pokrenute kontejnere:
			\begin{verbatim}
				docker ps -a
			\end{verbatim}
			\item Zaustavite kontejner koristeći njegov CONTAINER\_ID:
			\begin{verbatim}
				docker stop <CONTAINER_ID>
			\end{verbatim}
			Zamijenite \texttt{<CONTAINER\_ID>} s stvarnim ID-om kontejnera koji želite zaustaviti.
		\end{enumerate}

		\subsection*{Brisanje Docker Kontejnera}
		Za brisanje Docker kontejnera:
		\begin{enumerate}
			\item Navedite sve pokrenute kontejnere:
			\begin{verbatim}
				docker ps -a
			\end{verbatim}
			\item Obrišite kontejner koristeći njegov CONTAINER\_ID:
			\begin{verbatim}
				docker rm <CONTAINER_ID>
			\end{verbatim}
			Zamijenite \texttt{<CONTAINER\_ID>} s stvarnim ID-om kontejnera koji želite obrisati.
		\end{enumerate}

		\subsection*{Brisanje Docker Slike}
		Za brisanje Docker slike:
		\begin{enumerate}
			\item Navedite sve Docker slike:
			\begin{verbatim}
				docker images
			\end{verbatim}
			\item Obrišite sliku koristeći njegov IMAGE\_ID:
			\begin{verbatim}
				docker rmi <IMAGE_ID>
			\end{verbatim}
			Zamijenite \texttt{<IMAGE\_ID>} s stvarnim ID-om slike koju želite obrisati.
		\end{enumerate}

		\subsection*{Upravlanje bazom podataka}
			Starna baza podataka je već generirana i nalazi se u mapi \texttt{IzvorniKod/app/db}.
			Dakle nije potrebno ručno kreirati bazu podataka.
			U slučaju da želite jednu od operacija na bazi podataka, potrebno je zamijeniti vrijednost u .env datoteci.

			Ovo su sve moguće opcije za .env datoteku:
			MIGRATE=
			\begin{itemize}
				\item \texttt{true} - briše trenutnu bazu podataka i stvara novu \textbf{praznu} bazu podataka
				\item \texttt{reset} - briše trenutnu bazu podataka i učitava startnu bazu podataka sa startnim podacima
				\item \texttt{test} - koristi se kod pokretanja testova 
			\end{itemize}
	
		\subsection*{Vrijednosti .env datoteke}
			Ovdje su navedene sve mogućnosti za .env datoteku, za brzi setup aplikacije pogledajte \texttt{.env.example} datoteku.
			\begin{itemize}
				\item \texttt{ELEVEN\_VOICE\_KEY} - ključ za Eleven Voice API
				\item \texttt{GMAIL\_KEY} - ključ za Gmail API
				\item \texttt{MIGRATE} - vrijednost za upravljanje bazom podataka
				\item \texttt{TEST} - vrijednost za upravljanje testovima, \textbf{treba ostati false}
				\item \texttt{TESTMAIL} - email za testiranje, \textbf{uvijek mora biti prazno}.
				\item \texttt{PORT} - port na kojem će se pokrenuti aplikacija
			\end{itemize}
		
		\eject
