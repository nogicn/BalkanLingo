\chapter{Implementacija i korisničko sučelje}
		
		
		\section{Korištene tehnologije i alati}
			
		\subsection*{Komunikacija}
			Tijekom razvoja projekta koristili smo mnoge tehnologije kako bi olakšali i ubrzali komunikaciju. Većina interne projektne komunikacije odvijala se unutar privatnog servera u aplikaciji \textbf{\href{https://discord.com/}{Discord}}, dok je ostatak komunikacije bio preko \textbf{\href{https://www.microsoft.com/hr-hr/microsoft-teams/download-app}{MS Teams-a}}.
			Discord je aplikacija koja omogućava brzo postavljanje kanala za razgovor i brzo odvajanje logičkih jedinica projekta (front-end, back-end, dokumentacija).
			Za komunikaciju o projektu, zadatcima i projektnim obavijestima s asistenticom koristili smo \href{https://www.microsoft.com/hr-hr/microsoft-teams/download-app}{MS Teams}.


		\subsection*{Dokumentacija i verziranje aplikacije}
			Dokumentacija je napisana u sustavu \textbf{LaTeX}. \href{https://www.latex-project.org/}{LaTeX} je jezik koji omogućuje lagano pisanje dokumentacije kao i mogućnost brze iteracije i izmjena podataka. Za izradu grafova u projektu koristili smo \href{https://www.visual-paradigm.com/}{Visual Paradigm}. Verziranje aplikacije je bitan dio u procesu razvijanja aplikacije jer omogućava nadgledanje koji dio koda je kada dodan i mogućnost vraćanja u prošlost ako neki dio koda ne radi u budućnosti. Za verziranje koda koristili smo \textbf{\href{https://git-scm.com/}{Git}}, a za spremanje tih promjena koristili smo udaljeni repozitorij na \textbf{\href{https://github.com/}{GitHub-u}}.

		\subsection*{Aplikacija}
			Aplikacija je napisana u \textbf{\href{https://developer.mozilla.org/en-US/docs/Web/JavaScript}{JavaScript-u}} koristeći \textbf{\href{https://nodejs.org/en}{Node.js}} i \textbf{\href{https://www.npmjs.com/package/better-sqlite3}{SQLITE3}} bazu podataka. Node je okruženje za izvršavanje JavaScripta na bilo kojoj platformi izvan preglednika te se koristi kao backend server. Naša aplikacija nema odvojen front-end i back-end jer svu logiku izvršavamo na backendu u Node-u. Na backendu koristimo i \textbf{\href{https://expressjs.com/}{Express}} radni okvir koji nam olakšava pisanje logike HTTP zahtjeva. Za stvaranje HTML-a koji se onda šalje korisniku koristimo \href{https://ejs.co/}{ejs} šablonski jezik. Za SQLITE implementaciju koristimo \textbf{\href{https://www.npmjs.com/package/better-sqlite3}{better-sqlite3}} biblioteku koja nam omogućava brzo i jednostavno stvaranje baze podataka kao i bolje debugiranje problema oko upita.

		\subsection*{Razvojno okruženje}
			Za razvojno okruženje koristili smo \textbf{\href{https://code.visualstudio.com/}{Visual Studio Code}}. Visual Studio Code je besplatni izvorni kod editor koji je razvila tvrtka Microsoft. Za stiliziranje koda koristili smo \textbf{\href{https://prettier.io/}{Prettier}}. Prettier je alat koji nam omogućava da kod bude stilski konzistentan i da se ne moramo brinuti o stilu koda.

		\subsection*{Korisni linkovi}
			\begin{itemize}
				\item Discord: \url{https://discord.com/}
				\item Microsoft Teams: \url{https://www.microsoft.com/hr-hr/microsoft-teams/download-app}
				\item LaTeX: \url{https://www.latex-project.org/}
				\item Visual Paradigm: \url{https://www.visual-paradigm.com/}
				\item Git: \url{https://git-scm.com/}
				\item GitHub: \url{https://github.com/}
				\item JavaScript: \url{https://developer.mozilla.org/en-US/docs/Web/JavaScript}
				\item Node.js: \url{https://nodejs.org/en}
				\item Express: \url{https://expressjs.com/}
				\item EJS: \url{https://ejs.co/}
				\item Visual Studio Code: \url{https://code.visualstudio.com/}
				\item Prettier: \url{https://prettier.io/}
				\item Better SQLite3: \url{https://www.npmjs.com/package/better-sqlite3}
			\end{itemize}

				
			\eject 
		
	
		\section{Ispitivanje programskog rješenja}
			
			\textbf{\textit{dio 2. revizije}}\\
			
			 \textit{U ovom poglavlju je potrebno opisati provedbu ispitivanja implementiranih funkcionalnosti na razini komponenti i na razini cijelog sustava s prikazom odabranih ispitnih slučajeva. Studenti trebaju ispitati temeljnu funkcionalnost i rubne uvjete.}
	
			
			\subsection{Ispitivanje komponenti}
			\textit{Potrebno je provesti ispitivanje jedinica (engl. unit testing) nad razredima koji implementiraju temeljne funkcionalnosti. Razraditi \textbf{minimalno 6 ispitnih slučajeva} u kojima će se ispitati redovni slučajevi, rubni uvjeti te izazivanje pogreške (engl. exception throwing). Poželjno je stvoriti i ispitni slučaj koji koristi funkcionalnosti koje nisu implementirane. Potrebno je priložiti izvorni kôd svih ispitnih slučajeva te prikaz rezultata izvođenja ispita u razvojnom okruženju (prolaz/pad ispita). }
			
			
			
			\subsection{Ispitivanje sustava}
			
			 \textit{Potrebno je provesti i opisati ispitivanje sustava koristeći radni okvir Selenium\footnote{\url{https://www.seleniumhq.org/}}. Razraditi \textbf{minimalno 4 ispitna slučaja} u kojima će se ispitati redovni slučajevi, rubni uvjeti te poziv funkcionalnosti koja nije implementirana/izaziva pogrešku kako bi se vidjelo na koji način sustav reagira kada nešto nije u potpunosti ostvareno. Ispitni slučaj se treba sastojati od ulaza (npr. korisničko ime i lozinka), očekivanog izlaza ili rezultata, koraka ispitivanja i dobivenog izlaza ili rezultata.\\ }
			 
			 \textit{Izradu ispitnih slučajeva pomoću radnog okvira Selenium moguće je provesti pomoću jednog od sljedeća dva alata:}
			 \begin{itemize}
			 	\item \textit{dodatak za preglednik \textbf{Selenium IDE} - snimanje korisnikovih akcija radi automatskog ponavljanja ispita	}
			 	\item \textit{\textbf{Selenium WebDriver} - podrška za pisanje ispita u jezicima Java, C\#, PHP koristeći posebno programsko sučelje.}
			 \end{itemize}
		 	\textit{Detalji o korištenju alata Selenium bit će prikazani na posebnom predavanju tijekom semestra.}
			
			\eject 
		
		
		\section{Dijagram razmještaja}
			
			\textbf{\textit{dio 2. revizije}}
			
			 \textit{Potrebno je umetnuti \textbf{specifikacijski} dijagram razmještaja i opisati ga. Moguće je umjesto specifikacijskog dijagrama razmještaja umetnuti dijagram razmještaja instanci, pod uvjetom da taj dijagram bolje opisuje neki važniji dio sustava.}
			
			\eject 
		
		\section{Upute za puštanje u pogon}
		
		\subsection*{Instaliranje nodejs-a i npm-a}
		\begin{enumerate}
			\item Preuzmite nodejs instalaciju s \textbf{\href{https://nodejs.org/en/download/}{službene stranice}}.
			\item Pokrenite instalaciju.
			\item Provjerite je li nodejs uspješno instaliran:
			\begin{verbatim}
				node -v
			\end{verbatim}
			\item Provjerite je li npm uspješno instaliran:
			\begin{verbatim}
				npm -v
			\end{verbatim}
		\end{enumerate}

		\subsection*{Ručno Instaliranje}
		Za ručno pokretanje aplikacije potrebna je Node.js verzija 20 ili novija.
		\begin{enumerate}
			\item Promijenite direktorij u mapu "app":
			\begin{verbatim}
				dir IzvorniKod/app 
			\end{verbatim}
			\item Instalirajte projektne module:
			\begin{verbatim}
				npm install
			\end{verbatim}
			\item Pokrenite aplikaciju:
			\begin{verbatim}
				npm run start
			\end{verbatim}
			Aplikacija se pokreće na portu 3000.
		\end{enumerate}
		
		\subsection*{Instaliranje Docker-a}
		\begin{enumerate}
			\item Preuzmite Docker instalaciju s \textbf{\href{https://docs.docker.com/get-docker/}{službene stranice}}.
			\item Pokrenite instalaciju.
			\item Provjerite je li Docker uspješno instaliran:
			\begin{verbatim}
				docker -v
			\end{verbatim}
		\end{enumerate}

		\subsection*{Postavljanje Docker-a}
		Za pokretanje aplikacije pomoću Docker-a:
		\begin{enumerate}
			\item Promijenite direktorij u mapu "IzvorniKod":
			\begin{verbatim}
				cd IzvorniKod
			\end{verbatim}
			\item Izgradite Docker sliku pod nazivom "balkan-lingo":
			\begin{verbatim}
				docker build -t balkan-lingo .
			\end{verbatim}
			\item Pokrenite Docker kontejner i mapirajte port 3000 iz kontejnera na vaš poslužiteljski port:
			\begin{verbatim}
				docker run -p 3000:3000 balkan-lingo
			\end{verbatim}
			Za pokretanje kontejnera u pozadini koristite:
			\begin{verbatim}
				docker run -d -p 3000:3000 balkan-lingo
			\end{verbatim}
		\end{enumerate}
		
		\subsection*{Zaustavljanje Docker Kontejnera}
		Za zaustavljanje Docker kontejnera:
		\begin{enumerate}
			\item Navedite sve pokrenute kontejnere:
			\begin{verbatim}
				docker ps -a
			\end{verbatim}
			\item Zaustavite kontejner koristeći njegov CONTAINER\_ID:
			\begin{verbatim}
				docker stop <CONTAINER_ID>
			\end{verbatim}
			Zamijenite \texttt{<CONTAINER\_ID>} s stvarnim ID-om kontejnera koji želite zaustaviti.
		\end{enumerate}

		\subsection*{Brisanje Docker Kontejnera}
		Za brisanje Docker kontejnera:
		\begin{enumerate}
			\item Navedite sve pokrenute kontejnere:
			\begin{verbatim}
				docker ps -a
			\end{verbatim}
			\item Obrišite kontejner koristeći njegov CONTAINER\_ID:
			\begin{verbatim}
				docker rm <CONTAINER_ID>
			\end{verbatim}
			Zamijenite \texttt{<CONTAINER\_ID>} s stvarnim ID-om kontejnera koji želite obrisati.
		\end{enumerate}

		\subsection*{Brisanje Docker Slike}
		Za brisanje Docker slike:
		\begin{enumerate}
			\item Navedite sve Docker slike:
			\begin{verbatim}
				docker images
			\end{verbatim}
			\item Obrišite sliku koristeći njegov IMAGE\_ID:
			\begin{verbatim}
				docker rmi <IMAGE_ID>
			\end{verbatim}
			Zamijenite \texttt{<IMAGE\_ID>} s stvarnim ID-om slike koju želite obrisati.
		\end{enumerate}

		\subsection*{Upravlanje bazom podataka}
			Starna baza podataka je već generirana i nalazi se u mapi \texttt{IzvorniKod/app/db}.
			Dakle nije potrebno ručno kreirati bazu podataka.
			U slučaju da želite jednu od operacija na bazi podataka, potrebno je zamijeniti vrijednost u .env datoteci.

			Ovo su sve moguće opcije za .env datoteku:
			MIGRATE=
			\begin{itemize}
				\item \texttt{true} - briše trenutnu bazu podataka i stvara novu \textbf{praznu} bazu podataka
				\item \texttt{reset} - briše trenutnu bazu podataka i učitava startnu bazu podataka sa startnim podacima
				\item \texttt{test} - koristi se kod pokretanja testova 
			\end{itemize}
	
		\subsection*{Vrijednosti .env datoteke}
			Ovdje su navedene sve mogućnosti za .env datoteku, za brzi setup aplikacije pogledajte \texttt{.env.example} datoteku.
			\begin{itemize}
				\item \texttt{ELEVEN\_VOICE\_KEY} - ključ za Eleven Voice API
				\item \texttt{GMAIL\_KEY} - ključ za Gmail API
				\item \texttt{MIGRATE} - vrijednost za upravljanje bazom podataka
				\item \texttt{TEST} - vrijednost za upravljanje testovima, \textbf{treba ostati false}
				\item \texttt{TESTMAIL} - email za testiranje, \textbf{uvijek mora biti prazno}.
				\item \texttt{PORT} - port na kojem će se pokrenuti aplikacija
			\end{itemize}
		
		\eject
