\chapter{Opis projektnog zadatka}

		Cilj ovog projekta je razviti programsku potporu za web aplikaciju "BalkanLingo" koja omogućuje koristnicima da uče jezike.
		Naša aplikacija će omogućiti korisnicima da uče jezike na zabavan i interaktivan način kroz web aplikaciju .
		Aplikacija ima više načina učenja, a to su: 
		\begin{packed_item}
			\item učenje upitom prijevoda
			\item izgovor riječi
			\item slušanje riječi
		\end{packed_item}

		Učenje u ovoj aplikaciji se radi pomoću koncepta "spaced repetition" koji je osmislio Piotr Wozniak. Cilj spaced repetitiona je da se korisniku prikazuje riječ koju je korisnik već naučio, ali u različitim vremenskim intervalima.
		U našoj aplikaciji korisniku se prikazuje riječ koju je korisnik već naučio, ali u vremenskim intervalima od 1,3,5,7,15,30 dana. Ako čovjek točno odgovori na pitanje, onda se taj interval povećava, a ako odgovori netočno, onda se interval vraća na 1 dan.
		Kod učenja upitom prijevoda opisanog u sljedećem poglavlju, korisniku se prikazuje riječ na jednom jeziku, a korisnik treba odabrati točan prijevod te riječi na drugom jeziku.
		\\

		\textbf{Učenje upitom prijevoda} je način učenja u kojem korisniku se prikazuju 4 riječi na jednom jeziku, a korisnik treba odabrati točan prijevod te riječi na drugom jeziku.
		Na ovaj način korisnik uči riječi i njihove prijevode. Aplikacija koja radi na sličan način za ovaj tip je Kahoot, koji nudi 4 odgovora od kojih je samo jedan točan na postavljeno pitanje. U našem slučaju mi ne postavljamo pitanja nego tražimo prijevod riječi.
		Osim što su mu ponuđene 4 riječi, korisniku se prikazuje i primjer rečenice na oba jezika kako bi korisnik mogao lakše odabrati točan prijevod.
		\begin{figure}[H]
			\centering
			\includegraphics[width=0.8\linewidth]{slike/Kahoot.png}
			\caption{Primjer odabira točnog odgovora sa Kahoota}
			\label{fig:rijecnik}
		\end{figure}
		
		\textbf{Izgovor riječi} je način učenja u kojem korisniku se prikazuje riječ na jednom jeziku, a korisnik onda dobiva povratnu informaciju koliko je točan njegov izgovor. Aplikacjia koristi vrlo napredan sustav za prepoznavanje govora koji vraća postotak točnosti izgovora.
		\\
		\\
		\textbf{Slušanje riječi} je način učenja u kojem korisniku se prikazuje riječ na jednom jeziku, a korisnik mora upisati kako se ta riječ izgovara na drugom jeziku. Naša aplikacija koristi API koji vraća izgovor riječi na odabranom jeziku.
		\begin{figure}[H]
			\centering
			\includegraphics[width=0.4\linewidth]{slike/Duolingo.png}
			\caption{Primjer slušanja riječi iz Duolinga}
			\label{fig:rijecnik}
		\end{figure}

		\textbf{Registracija} je proces u kojem korisnik unosi svoje podatke kako bi mogao koristiti aplikaciju.
		Korisnik mora unijeti svoje ime, prezime, korisničko ime, e-mail adresu.
		Ako je registracija uspješna, korisnik na email dobiva random password koji mora promijeniti prilikom prvog logina.
		\\
		\\
		\textbf{Prijava} je proces u kojem korisnik unosi svoje korisničko ime i lozinku kako bi mogao koristiti aplikaciju.
		Ako korisnik postoji, onda se prikazuje dashboard, a ako ne postoji, onda se prikazuje poruka da korisnik ne postoji
		\\
		\\
		\textbf{Dashboard} je stranica na kojoj se nalaze riječnici koje je korisnik dosad učio, ako nije učio niti jedan rječnik, onda se prikazuje prazna stranica sa jednom ikonom koja otvara stranicu za dodavanje novog rječnika koji želi učiti.
		\\
		\\
		\textbf{Rječnik} je skup riječi koje korisnik želi učiti.
		Rječnik se sastoji od imena rječnika, jezika na kojem je rječnik, riječi u oba jezika i primjer rečenice na oba jezika.
		\\
		\\
		Aplikacija ima dva tipa računa i to su:
		\begin{packed_item}
			\item \textbf{Korisnik} 
			\item \textbf{Administrator}
		\end{packed_item}

		\textbf{Korisnik} ima mogućnost učenja rječnika i jezika koristeći sva tri načina učenja koja se rotiraju.
		Kada se korisnik ulogira u aplikaciju, onda se prikazuje dashboard sa svim riječnicima koje je korisnik dosad učio ili dodao.
		Korisnik ima mogućnost dodavanja novih rječnika koje želi učiti iz postojeće baze podataka.
		Kada korisnik točno odgovori na pitanje, onda se ta riječ stavlja u pool riječi koji ima odstupanje od 1,3,5,7,15,30 dana.
		To odstupanje se povečava svaki put kada korisnik točno odgovori na pitanje, ako odgovori netočno, onda se odstupanje vraća na 1 dan.
		Ako korisnik tijekom rješavanja prijeđe na novi uređaj ili zatvori aplikaciju, onda se njegovo zadnje pitanje sprema u bazu podataka.
		Korisnik ima mogućnost resetiranja lozinke.
		\\
		\\
		\textbf{Administrator} ima mogućnost dodavanja, brisanja i izmjenivanja novih rječnika, jezika i riječi u bazu podataka.
		Riječi se dodaju pomoću besplatnih javnih API servisa koji vraćaju riječi na odabranom jeziku. Kada ih administrator doda, može editirati sve o njima.
		Dakle može promijeniti riječ, prijevod i primjer rečenice u oba jezika.
		Po potrebi može resetirati lozinku korisnika.
		Administrator ne može učiti rječnike jer mu se na dashboardu prikazuju svi riječnici koji postoje u bazi podataka.
		Administrator može dodati i nove administratore. Kada administrator obriše rječnik, onda se i svim korisnicima brišu sve riječi koje su pripadale tom rječniku.
		\\
		\\

		\eject
		
	