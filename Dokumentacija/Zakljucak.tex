\chapter{Zaključak i budući rad}
		
		Web aplikacija BalkanLingo rezultat je timskog rada u sklopu kolegija Programsko inženjerstvo. Dobili smo zadatak izraditi edukativnu platformu koja korisnicima pomaže u usvajanju stranih jezika. Kao i kod svakog projekta, tim je zadatak obogatio svojim idejama i prilagodio ga vlastitoj percepciji procesa učenja te smo se zbog toga oslanjali na igrifikaciju i interaktivnost.
		\newline
		\\
		Zadatak nas je vodio kroz sve faze razvijanja programske potpore i upoznao nas s potencijalnim izazovima i preprekama koji se u navedenom procesu javljaju. Uz sam razvoj programske potpore, tim je morao naučiti skladno i produktivno raditi, zbog čega je veliki naglasak bio i na organizaciji posla.
		\newline
		\\
		Projekt se odvijao u dvije faze. Tijekom prve faze upoznavali smo se s novim tehnologijama koje smo koristili pri izradi prototipa, a tijekom druge faze izrađivali smo konačni proizvod. U nastavku je rezime onoga što smo napravili i naučili.
		\newline
		\\
		\textbf{Prva faza} prvenstveno je bila namijenjena upoznavanju. Osim što smo istraživali nove tehnologije, neki su se članovi prvi put upoznali u sklopu projekta. Razgovorom i diskusijom uspjeli smo doći do popisa tehnologija koji odgovara svima. Svi članovi bili su upoznati s radnim okruženjem Express, ali i tehnologijom EJS. 
		\newline
		\\
		Raspodjela uloga također nije bila komplicirana jer su članovi tima bili kompetentni upravo u onim poljima u kojima su htjeli raditi. Voditelj tima konkretizirao je uloge namijenjene ostalih šest članova i imali smo veliki početni sastanak u kojem smo svi zajedno razvili koncept aplikacije. Tada smo zajedno, razgovarajući o aplikaciji, pomogli članici tima koja se bavila korisničkim sučeljem da napravi grube skice. Na taj način svi smo dobili bolju ideju što točno radimo i tko bi što mogao implementirati.
		\newline
		\\
		Odlučili smo da će sve interakcije sa poslužiteljem rezultirati u HTML odgovoru. Ova nas je odluka ubrzala te smo u sklopu prototipa predali i više od traženih funkcionalnosti. Jedan od razloga svakako je činjenica da nismo duplicirali logiku za predstavljanje rezultata (što bi se svakako dogodilo da smo koristili React ili Vue). Mana ovog pristupa jest osvježavanje stranice pri svakoj promjeni, ali to smo odlučili ručno riješiti poslije predaje prototipa.
		\newline
		\\
		Izbor baze podataka, zahvaljujući kompetentnosti članova tima, bio je vrlo jednostavan. SQLITE bio je savršen izbor za naš prototip jer nema potrebe niti za postavljanjem baze podataka na računalima svih članova tima niti za sinkronizacijom stanja baze. Iako smo se prvo orijentirali na paket sqlite3, kasnije smo otkrili better-sqlite3 koji je bio daleko jednostavniji i jasniji.
		\newline
		\\
		 Za dokumentiranje projekta koristili smo alate LaTeX i VisualParadigm. Svi članovi ažurno su pisali svoje dijelove dokumentacije i međusobno smo se ispravljali kada je to bilo potrebno. Voditelj projekta inzistirao je na tome da napravimo dobar kostur za prototip jer će nam to omogućiti da rano uočimo implementacijske probleme. Zaista, dodavanje funkcionalnosti za konačni proizvod zbog toga je bilo jednostavnije i efikasnije.
		 \newline
		\\
		\textbf{Druga faza} većinom je bila orijentirana na implementaciju svih funkcionalnosti. Nismo mijenjali prvotno odabrane tehnologije jer niti jedna nije predstavljala problem tijekom neformalnih, a kasnije i formalnih ispitivanja. SQLITE za naše je potrebe bio dovoljno efikasan i radio je bez sinkronizacijskih problema. 
		\newline
		\\
		Uvrstili smo i jednu novu tehnologiju, HTMX. Radi se o biblioteci koja omogućava interakciju klijenta i servera bez implementacije dodatnog JavaScript koda. Ovu smo tehnologiju koristili kod automatskog podešavanja riječi, pretraživanja rječnika i u tranziciji između načina učenja. Iako smo u ovoj fazi intenzivno programirali, brzina napretka projekta bila je i više no zadovoljavajuća zbog dobrog odabira tehnologija.
		\newline
		\\
		I prije formalne faze ispitivanja napravili smo prve testove kako bismo provjerili jesmo li dodali provjere autentičnosti na bitne administratorske rute. Kasnije smo dodali testove čija je svrha provjera sadržaja baze podataka nakon neke akcije koja je trebala stvoriti i/ili modificirati određeni dio same baze. 
		\newline
		\\
		Iako dokumentiranje u drugoj fazi razvoja aplikacije nije zahtijevalo toliko raspisivanja i definiranja, pravovremeno smo morali ispravljati dijelove dokumentacije napisane u prvoj fazi jer smo u određenim aspektima mijenjali pristup problemu.
		\newline
		\\
	   \indent 
	   Naš projekt zadovoljava sve kriterije zadatka, odnosno implementirane su sve potrebne funkcionalnosti. Svi članovi tima zadovoljni su s rezultatom projekta i naučili su ponešto i izvan onoga što ih je najviše zanimalo (čime su se primarno bavili prije i tijekom projekta). Svakako smo kao krajnji rezultat, uz aplikaciju i njenu popratnu dokumentaciju, dobili iskustvo timskog rada.
		 \eject 